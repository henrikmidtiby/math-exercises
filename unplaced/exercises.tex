\documentclass{article}
\usepackage{tekvideoexercises}

\begin{document}
\exercisename{Random exercises}

\tableofcontents
\newpage

\begin{exercise}{Differentiering 1-1}
% Test comment.

Bestem $\frac{d}{dx} x^2$.

\answerbox{2x}

\hint

Benyt formlen
\[
\frac{d}{dx} x^n = n \cdot x^{n - 1}
\]

\hint

Sæt $n = 2$
\[
\frac{d}{dx} x^2 = 2 \cdot x^{2 - 1} = 2 x
\]

\end{exercise}

\begin{exercise}{Differentiering 1-2}

Bestem $\frac{d}{dx} x^3$.

\answerbox{3x^2}

\hint

Benyt formlen
\[
\frac{d}{dx} x^n = n \cdot x^{n - 1}
\]

\hint

Sæt $n = 3$
\[
\frac{d}{dx} x^3 = 3 \cdot x^{3 - 1} = 3 x^2
\]

\end{exercise}


\begin{exercise}{Differentiering 1-3}
Bestem $\frac{d}{dx} \left( x \cdot \sin(x) \right)$.

\answerbox{\sin(x) + x \cdot \cos(x)}

\hint
Benyt produkt reglen
\[
\frac{d}{dx} \left( f(x) \cdot g(x) \right) = \frac{d}{dx} \left( f(x) \right) \cdot g(x) + f(x) \cdot \frac{d}{dx} \left( g(x) \right)
\]

\hint
Sæt $f(x) = x$ og $g(x) = \sin(x)$
\[
\frac{d}{dx} \left( x \cdot \sin(x) \right) = \frac{d}{dx} \left( x \right) \cdot \sin(x) + x \cdot \frac{d}{dx} \left( \sin(x) \right)
\]

\hint
Differentier delled.
\[
\frac{d}{dx} \left( x \cdot \sin(x) \right) = 1 \cdot \sin(x) + x \cdot \cos(x)
\]

\hint
Der forenkles til 
\[
\sin(x) + x \cdot \cos(x)
\]

\end{exercise}



\begin{exercise}{Differentiering 1-4}
Bestem $\frac{d}{dx} \left( e^x \cdot \sin(x) \right)$.

\answerbox{e^x \cdot \sin(x) + e^x \cdot \cos(x)}

\hint
Benyt produkt reglen
\[
\frac{d}{dx} \left( f(x) \cdot g(x) \right) = \frac{d}{dx} \left( f(x) \right) \cdot g(x) + f(x) \cdot \frac{d}{dx} \left( g(x) \right)
\]

\hint
Sæt $f(x) = e^x$ og $g(x) = \sin(x)$
\[
\frac{d}{dx} \left( e^x \cdot \sin(x) \right) = \frac{d}{dx} \left( e^x \right) \cdot \sin(x) + e^x \cdot \frac{d}{dx} \left( \sin(x) \right)
\]

\hint
Differentier delled.
\[
\frac{d}{dx} \left( x \cdot \sin(x) \right) = e^x \cdot \sin(x) + e^x \cdot \cos(x)
\]
\end{exercise}



\begin{exercise}{Differentiering 1-5}
Bestem $\frac{d}{dx} \left( \sin(e^x) \right)$.

\answerbox{e^x \cdot \cos(e^x)}

\hint
Benyt kæde reglen
\[
\frac{d}{dx} f(g(x)) = f'(g(x)) \cdot g'(x)
\]

\hint
Sæt $f(x) = \sin(x)$ og $g(x) = e^x$ og bestem $f'(x)$ og $g'(x)$.
\[
f'(x) = \cos(x) \qquad g'(x) = e^x
\]

\hint
Indsæt i kædereglen
\[
\frac{d}{dx} \sin(e^x) = \cos(e^x) \cdot e^x
\]
\end{exercise}


\begin{exercise}{Differentiering 1-6}
% Adams 3.5.19
Bestem $\frac{d}{dx} \arcsin\left( \frac{2x-1}{3} \right)$.

\answerbox{(2+x-x^2)^{-1/2}}


\hint
\[
\frac{d}{dx} \arcsin(x) = \frac{1}{\sqrt{1 - x^ 2}}
\]


\hint
Benyt kæde reglen
\[
\frac{d}{dx} f(g(x)) = f'(g(x)) \cdot g'(x)
\]

\hint
Sæt $f(x) = \arcsin(x)$ og $g(x) = \frac{2x - 1}{3}$.

\hint
Bestem $f'(x)$ og $g'(x)$.
\[
f'(x) = \frac{1}{\sqrt{1 - x^ 2}} \qquad g'(x) = 2/3
\]

\hint
Indsæt i kædereglen
\[
\frac{d}{dx} \arcsin\left( \frac{2x-1}{3} \right) = \frac{1}{\sqrt{1 - \left( \frac{2x - 1}{3} \right)^ 2}} \cdot \frac{2}{3}
\]

\hint
Fjern parentesen under kvadratrodstegnet
\[
\frac{1}{\sqrt{1 - \left( \frac{2x - 1}{3} \right)^ 2}} \cdot \frac{2}{3}
= \frac{1}{\sqrt{\frac{9 - 4x^2 + 4x - 1}{9}}} \cdot \frac{2}{3}
\]

\hint
Lidt forenklinger
\[
\frac{1}{\sqrt{\frac{9 - 4x^2 + 4x - 1}{9}}} \cdot \frac{2}{3}
= \frac{2}{\sqrt{8 - 4x^2 + 4x}}
= \frac{1}{\sqrt{2 - x^2 + x}}
\]
\end{exercise}




\begin{exercise}{Differentiering 1-7}
% Adams 3.5.20
Differentier udtrykket $\arctan\left( a x + b \right)$ med hensyn til $x$.

\answerbox{\frac{a}{1+(a \cdot x +b)^2}}

\hint
\[
\frac{d}{dx} \arctan(x) = \frac{1}{1 + x^ 2}
\]


\hint
Benyt kæde reglen
\[
\frac{d}{dx} f(g(x)) = f'(g(x)) \cdot g'(x)
\]

\hint
Sæt $f(x) = \arctan(x)$ og $g(x) = a \cdot x + b$.

\hint
Bestem $f'(x)$ og $g'(x)$.
\[
f'(x) = \frac{1}{1 + x^ 2} \qquad g'(x) = a
\]

\hint
Indsæt i kædereglen
\[
\frac{d}{dx} \arctan\left( a \cdot x + b \right) = \frac{1}{1 + \left( a \cdot x + b \right)^2} \cdot a
\]

\hint
Lidt forenklinger
\[
\frac{a}{1 + \left( a \cdot x + b \right)^2}
\]
\end{exercise}


\begin{exercise}{Linearisering 1}
% Adams 4.9.7
Lineariser nedenstående funktion omkring punktet $x = \pi$.
\[
\sin(x)
\]

\answerbox{\pi - x}

\hint
Lineariseringen af en funktion $f(x)$
omkring punktet $x_0$ er givet ved
\[
f(x_0) + f'(x_0) \cdot (x - x_0).
\]

\hint
Vi anvender $f(x) = \sin(x)$ og $x_0 = \pi$.

\hint
$f'(x)$ bestemmes
\[
f'(x) = \cos(x)
\]

\hint
Værdierne fra opgaven indsættes i udtrykket for lineariseringen.
\[
\sin(\pi) + \cos(\pi) \cdot (x - \pi)
\]

\hint
Der forsimples
\[
\sin(\pi) = 0 \qquad 
\cos(\pi) = -1
\]
og indsættes i udtrykket
\[
(0) + (-1) \cdot (x - \pi) = 
\pi - x
\]

\end{exercise}


\begin{exercise}{Linearisering 2}
% Adams 4.9.19
Benyt en passende linearisering til at approksimere
\[
\cos(49^\circ)
\]
angiv resultatet med fire decimalaer i boksen herunder.
\answerbox{0.6577}

\hint
Først skal vi vælge den funktion der skal approksimeres
og det punkt som der skal bruges som udgangspunkt for 
approksimationen.
For at vi kan bestemme approksimationen skal vi 
kende funktionsværdien og dens afledte i punktet.

\hint
Vi vælger funktionen $f(x) = \cos(x)$ og at approksimere den 
omkring $x_0 = 45^\circ = \pi / 4$.
Så kan vi nemlig godt finde værdierne af $f(\pi / 4)$ og $f'(\pi / 4)$.
\[
f(\pi / 4) = \cos(\pi / 4) = \frac{1}{\sqrt{2}}
\]
\[
f'(\pi / 4) = -\sin(\pi / 4) = \frac{-1}{\sqrt{2}}
\]

\hint
Lineariseringen opskrives
\[
L(x) = \frac{1}{\sqrt{2}} - \frac{1}{\sqrt{2}} \cdot (x - \pi / 4)
\]

\hint
Værdien i punktet bestemmes
\[
L(49 \cdot \pi / 180) = \frac{1}{\sqrt{2}} - \frac{1}{\sqrt{2}} \cdot (49 \cdot \pi / 180 - \pi / 4) 
\]
det forsimples til 
\[
\frac{1}{\sqrt{2}} - \frac{1}{\sqrt{2}} \cdot \pi / 45
\]
som vha. lommeregner bestemmes til 
\[
0.657741 \simeq 0.6577
\]

\hint
For at vurdere fejlen på approksimationen, ser på anden 
ordens taylor udviklingen (en orden højere end den 
anvendte linearisering).
\[
f(x_0) 
	+ \frac{f'(x_0)}{1!} \cdot (x - x_0) 
	+ \frac{f''(x_0)}{2!} \cdot (x - x_0)^2
\]
Fejlen er relateret til det sidste led i det ovenstående udtryk.
Hvor $f''(x_0)$ udskiftes med $f''(c)$ hvor $c \in [x_0; x]$.
\[
\textrm{fejl} = \frac{f''(c)}{2!} \cdot (x - x_0)^2
\]

\hint 
Antager man at $c = \pi/4$, fåes følgende fejlvurdering
\[
\textrm{fejl} = \frac{f''(\pi/4)}{2!} \cdot (49 \cdot \pi / 180 - \pi / 4)^2 = 
-0.00172318\]

\hint
Sammenligning mellem reel værdi, approksimeret værdi og estimeret maksimal fejl.
\[
\cos(49 \cdot \pi/180) = 0.656059
\]
\[
L(49 \cdot \pi/180) = 0.657741
\]
\[
\cos(49 \cdot \pi/180) - L(49 \cdot \pi/180) = -0.001682
\]

\end{exercise}


\begin{exercise}{Taylor polynomier 1}

Lav et fjerde ordens Taylor polynomie for $f(x) = e^{-x}$
omkring $x = 0$.

\answerbox{1 - x + \frac{x^2}{2} - \frac{x^3}{6} + \frac{x^4}{24}}

\hint
Et fjerde ordens Taylor polynomie for en funktion $f(x)$
omkring $x_0$ er givet ved
\[
f(x_0) + \frac{f'(x_0)}{1!} \cdot (x - x_0) + \frac{f''(x_0)}{2!} \cdot (x - x_0)^2 + \frac{f'''(x_0)}{3!} \cdot (x - x_0)^3 + \frac{f''''(x_0)}{4!} \cdot (x - x_0)^4
\]

\hint
Bestem $f(x_0) = f(0)$
\[
f(0) = e^{-0} = 1
\]

\hint
Bestem $f'(x_0) = f'(0)$
\[
f'(0) = \frac{d}{dx} e^{-x} \Big|_{x = 0} = -e^{-0} = -1
\]

\hint
Bestem $f''(x_0) = f''(0)$
\[
f''(0) = \frac{d^2}{dx^2} e^{-x} \Big|_{x = 0} = e^{-0} = 1
\]

\hint
Bestem $f'''(x_0) = f'''(0)$
\[
f'''(0) = \frac{d^3}{dx^3} e^{-x} \Big|_{x = 0} = -e^{-0} = -1
\]

\hint
Bestem $f''''(x_0) = f''''(0)$
\[
f''''(0) = \frac{d^4}{dx^4} e^{-x} \Big|_{x = 0} = e^{-0} = 1
\]

\hint
Sæt værdierne ind i opskriften givet i det første hint.
\[
1 + \frac{-1}{1!} \cdot (x - 0) + \frac{1}{2!} \cdot (x - 0)^2 + \frac{-1}{3!} \cdot (x - 0)^3 + \frac{1}{4!} \cdot (x - 0)^4
\]

\hint
Forenkel udtrykket
\[
1 - x + \frac{x^2}{2} - \frac{x^3}{6} + \frac{x^4}{24}
\]

\end{exercise}


\begin{exercise}{Taylor polynomier 2}

Lav et treje ordens Taylor polynomie for $f(x) = \sqrt{x}$
omkring $x = 4$.

\answerbox{1 + \frac{x - 4}{4} - \frac{(x - 4)^2}{64} + \frac{(x - 4)^3}{512}}

\hint
Et trejde ordens Taylor polynomie for en funktion $f(x)$
omkring $x_0$ er givet ved
\[
f(x_0) + \frac{f'(x_0)}{1!} \cdot (x - x_0) + \frac{f''(x_0)}{2!} \cdot (x - x_0)^2 + \frac{f'''(x_0)}{3!} \cdot (x - x_0)^3
\]

\hint
Bestem $f(x_0) = f(4)$
\[
f(4) = \sqrt{4} = 2
\]

\hint
Bestem $f'(x_0) = f'(4)$
\[
f'(4) = \frac{d}{dx} x^{1/2} \Big|_{x = 4} = \frac{1}{2} \cdot x^{-1/2} \Big|_{x = 4} = \frac{1}{4}
\]

\hint
Bestem $f''(x_0) = f''(4)$
\[
f''(4) = \frac{d}{dx} \left( \frac{1}{2} \cdot x^{-1/2} \right) \Big|_{x = 4} = \frac{1}{2} \cdot \frac{-1}{2} \cdot x^{-3/2} \Big|_{x = 4} = \frac{-1}{4} \cdot \frac{1}{8} = \frac{-1}{32}
\]

\hint
Bestem $f'''(x_0) = f'''(0)$
\[
f'''(4) = \frac{d}{dx} \left( \frac{-1}{4} \cdot x^{-3/2} \right) \Big|_{x = 4} = \frac{-1}{4} \cdot \frac{-3}{2} \cdot x^{-5/2} \Big|_{x = 4} = \frac{3}{8} \cdot \frac{1}{32} = \frac{3}{256} 
\]

\hint
Sæt værdierne ind i opskriften givet i det første hint.
\[
1 + \frac{1/4}{1!} \cdot (x - 4) + \frac{-1 / 32}{2!} \cdot (x - 4)^2 + \frac{3 / 256}{3!}  \cdot (x - 4)^3
\]

\hint
Forenkel udtrykket
\[
1 + \frac{x - 4}{4} - \frac{(x - 4)^2}{64} + \frac{(x - 4)^3}{512}
\]

\end{exercise}

\begin{exercise}{Taylor polynomier 3}

Bestem et første ordens taylor polynomie for $f(x) = x^3 - x + 1$
omkring $x_0 = -1$.

\answerbox{1 + (x + 1)}

\hint
Et første ordens taylor polynomie (dvs. en linearisering) 
er givet ved
\[
p_1(x) = f(x_0) + \frac{f'(x_0)}{1!} \cdot (x - x_0)
\]


\hint
Bestem først $f'(x)$
\[
f'(x) = 3x^2 - 1
\]

\hint
Bestem $f(x_0) = f(-1) = 1$.

\hint
Bestem $f'(x_0) = f'(-1) = 2$.

\hint
Sæt det hele sammen
\[
p_1(x) = 1 + \frac{2}{1!} \cdot \left(x - (-1)\right)
\]

\hint
Forenkel udtrykket
\[
p_1(x) = 1 + (x + 1)
\]


\end{exercise}


\begin{exercise}{Taylor polynomier 4}

Bestem et anden ordens taylor polynomie for $f(x) = x^4 - 7x$
omkring $x_0 = 2$.

\answerbox{2 + 25 \cdot (x - 2) + 24 \cdot (x - 2)^2}

\hint
Et anden ordens taylor polynomie 
er givet ved
\[
p_2(x) = f(x_0) + \frac{f'(x_0)}{1!} \cdot (x - x_0)
 + \frac{f''(x_0)}{2!} \cdot (x - x_0)^2
\]


\hint
Bestem $f(x_0)$
\[
f(x_0) = f(2) = 2
\]

\hint
Bestem $f'(x)$ og $f'(x_0)$
\[
f'(x) = 4x^3-7 \qquad f'(x_0) = f'(2) = 25
\]

\hint
Bestem $f''(x)$ og $f''(x_0)$
\[
f''(x) = 12 x^2 \qquad f''(x_0) = f''(2) = 48
\]

\hint
Sæt det hele sammen
\[
p_2(x) = 2 + \frac{25}{1!} \cdot \left(x - 2\right) + \frac{48}{2!} \cdot \left(x - 2\right)^2
\]

\hint
Forenkel udtrykket
\[
p_2(x) = 2 + 25 \cdot \left(x - 2\right) + 24 \cdot \left(x - 2\right)^2
\]


\end{exercise}


\begin{exercise}{Taylor polynomier 5}

Bestem et anden ordens taylor polynomie for $f(x) = 3 x^2 + x^3$
omkring $x_0 = 1$.
Benyt taylor polynomiet til at estimere $f(1.1)$.

\answerbox{4.96}

\hint
Et anden ordens taylor polynomie 
er givet ved
\[
p_2(x) = f(x_0) + \frac{f'(x_0)}{1!} \cdot (x - x_0)
 + \frac{f''(x_0)}{2!} \cdot (x - x_0)^2
\]


\hint
Bestem $f(x_0)$
\[
f(x_0) = f(1) = 3 \cdot 1^2 + 1^3 = 4
\]

\hint
Bestem $f'(x)$ og $f'(x_0)$
\[
f'(x) = 6 x + 3 x^2 \qquad f'(x_0) = f'(1) = 9
\]

\hint
Bestem $f''(x)$ og $f''(x_0)$
\[
f''(x) = 6 + 6x \qquad f''(x_0) = f''(1) = 12
\]

\hint
Sæt det hele sammen
\[
p_2(x) = 4 + \frac{9}{1!} \cdot \left(x - 1\right) + \frac{12}{2!} \cdot \left(x - 1\right)^2
\]

\hint
Forenkel udtrykket
\[
p_2(x) = 4 + 9 \cdot \left(x - 1\right) + 6 \cdot \left(x - 1\right)^2
\]

\hint
Estimer $f(1.1)$ vha. taylor polynomiet, ved at indsætte i $p_2(x)$.
Evaluer $p_2(1.1)$.

\hint
\[
p_2(1.1) = 4 + 9 \cdot \left(1.1 - 1\right) + 6 \cdot \left(1.1 - 1\right)^2 
 = 4 + 9 \cdot 0.1 + 6 \cdot 0.01 = 4.96
\]



\end{exercise}

\begin{exercise}{LHopital 1}
Bestem grænseværdien.
\[
\lim_{x \to 1} \frac{e^x - 1}{(x - 1)^2}
\]

Er svaret $\infty$ skrives der \emph{uendelig} i svarboksen.

\answerbox{uendelig}

\hint
Først undersøges det om vi må anvende L'Hopital.

\hint
Grænserne af tælleren og nævneren bestemmes.
\[
\lim_{x \to 1} (e^x - 1) = e^1 - 1 = e - 1 \neq 0 \\
\lim_{x \to 1} (x - 1)^2 = (1 - 1)^2 = 0
\]

\hint
Da de to grænser ikke begge er nul eller uendelig, må vi ikke anvende L'Hopitals regel. 

\hint
Vi ender i følgende situtation.
\[
\left[ \frac{e - 1}{0} \right]
\]

\hint
Hvor noget endeligt ($e - 1$) deles med noget der går mod nul.
Bemærk at $(x - 1)^2$ altid er større eller lig med nul. 

Derfor går hele værdien mod uendelig.
\[
\lim_{x \to 1} \frac{e^x - 1}{(x - 1)^2} = \infty
\]

\end{exercise}

\begin{exercise}{Separabel differentialligning 1}

Løs den separable differentialligning:
\[
\frac{dy}{dx} = \frac{1}{x} \cdot \frac{1}{y^2}
\]

Indskriv et udtryk for $y^3$.

\answerbox{3 \ln(x) + C}

\hint


Først konstateres det at ligningen er separabel, derefter 
ganges der igennem med $y^2$.
\[
y^2 \cdot \frac{dy}{dx} = \frac{1}{x}
\]

\hint
Så integreres der mht. $x$.
\[
\int y^2 \cdot \frac{dy}{dx} \, dx = \int \frac{1}{x} \, dx
\]

\hint
$dx$'erne går ud med hinanden på venstresiden.
\[
\int y^2 \, dy = \int \frac{1}{x} \, dx
\]

\hint
Integralerne løses.
\[
\frac{1}{3} y^3 + C_1 = \ln(x) + C_2
\]

\hint
Der ganges igennem med 3 og konstanterne samles i $C$.
\[
y^3 = 3 \ln(x) + C
\]

\end{exercise}


\begin{exercise}{Rumgeometri 1}

Find en enhedsvektor der er vinkelret på det plan
der indholder punkterne $(a, 0, 0)$, $(0, b, 0)$ og
$(0, 0, c)$.
Hvad er arealet af den trekant der er udspændt af 
disse tre punkter?

Intet svar...
\answerbox{0}

\hint
Kan løses ved at finde to vektorer der ligger i planet
og derefter finde en vektor der er vinkelret på 
disse to vektorer.

\hint
Krydsproduktet kan bruges til at finde en vektor der
er vinkelret på to andre vektorer.

\hint
Krydsproduktet kan også bruges til at bestemme arealet.

\end{exercise}


\begin{exercise}{Rumgeometri 2}

Find en enhedsvektor med en positiv $\mathbf{k}$ komponent, 
der er vinkelret på både
$2 \mathbf{i} - \mathbf{j} - 2 \mathbf{k}$
og 
$2 \mathbf{i} - 3\mathbf{j} + \mathbf{k}$.

Intet svar...
\answerbox{0}

\hint
Krydsproduktet kan bruges til at finde en vektor der
er vinkelret på to andre vektorer.

\hint
Gang vektoren med en skalar for at få det rigtige fortegn
på den ene koordinat.

\end{exercise}


\begin{exercise}{Rumgeometri 3}

Bestem ligningen for det plan der går igennem
punktet $(0, 2, -3)$ og er vinkelret på vektoren
$4 \mathbf{i} - \mathbf{j} - 2 \mathbf{k}$.

Intet svar...
\answerbox{0}

\hint
\[
\vec{n} \cdot \left( \vec{p} - \vec{p}_0 \right) = 0
\]

\end{exercise}


\begin{exercise}{Rumgeometri 4}

Bestem ligningen for det plan der går igennem
de tre punter: (1, 1, 0), (2, 0, 2) og (0, 3, 3). 

Intet svar...
\answerbox{0}

\hint
Kan løses ved at finde to vektorer der ligger i planet
og derefter finde en vektor der er vinkelret på 
disse to vektorer.

\hint
Krydsproduktet kan bruges til at finde en vektor der
er vinkelret på to andre vektorer.

\hint
\[
\vec{n} \cdot \left( \vec{p} - \vec{p}_0 \right) = 0
\]


\end{exercise}


\begin{exercise}{Rumgeometri 5}

Under hvilke geometriske forhold vil tre punkter i 
$\mathcal{R}^3$ ikke beskrive et unikt plan der går igennem
alle punkterne?
Hvordan kan denne betingelse skrives matematisk / algebraisk
ud fra positionsvektorerne $\mathbf{r}_1$, $\mathbf{r}_2$ og
$\mathbf{r}_3$ for de tre punkter?

Intet svar...
\answerbox{0}

\hint
Kan punkterne placeret uheldigt?

\hint
Hvad hvis de ligger på linje.

\hint
Lav en matrix med de koordinaterne til de tre punkter
i hver linje.
Hvis determinanten af matricen er 0 ligger punkterne på en linje, 
og planet er derfor ikke bestemt.

\end{exercise}


\end{document}
