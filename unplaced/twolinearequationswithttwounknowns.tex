\documentclass{article}
\usepackage[utf8]{inputenc}
\usepackage{todonotes}
\usepackage{graphicx}
\usepackage{amsmath}
\usepackage[colorlinks, linkcolor=blue, citecolor=blue, urlcolor=blue]{hyperref}


\newenvironment{exercise}[1]{\newpage\section{#1}}{}

\newcommand{\answerbox}[1]{\fbox{$#1$}}

\newenvironment{multichoice}[1][]{\par\noindent{}Parametre: #1\begin{itemize}}{\end{itemize}}
\newcommand{\itemtrue}{\item (true) }
\newcommand{\itemfalse}{\item (false) }

\newenvironment{answermatrix}{\(\begin{pmatrix}}{\end{pmatrix}\)}


\newcommand{\hint}{\subsection*{Hint}}

\begin{document}
\tableofcontents
\newpage

\begin{exercise}{Ligningssystemer 1-1}

Givet ligningerne \(2x + y = 3\) og \(x - y = 0\) bestem \(x\) og \(y\).

\(x = \) \answerbox{1}

\(y = \) \answerbox{1}

\hint

Læg de to ligninger sammen, det giver \(3x = 3\). Derefter kan \(x\) findes og vi har \(x = 1\).

\hint
Indsæt \(x = 1\) i den første ligning. 

\hint
Det giver \(2 \cdot 1 + y = 3\), der forenkles til \(2 + y = 3\), 
hvoraf \(y = 1\) findes.

\end{exercise}



\begin{exercise}{Ligningssystemer 1-2}

Givet ligningerne \(4x + y = 11\) og \(x - y = -1\) bestem \(x\) og \(y\).

\(x = \) \answerbox{2}

\(y = \) \answerbox{3}

\hint

Læg de to ligninger sammen, det giver \(5x = 10\). Derefter kan \(x\) findes og vi har \(x = 2\).

\hint
Indsæt \(x = 2\) i den første ligning. 

\hint
Det giver \(4 \cdot 2 + y = 11\), der forenkles til \(8 + y = 11\), 
hvoraf \(y = 3\) findes.

\end{exercise}



\begin{exercise}{Multiple choice test 1}

Er ligningen herunder linært i $x$, $y$ og $z$?
\[
2x + 3y + z = 2
\]

\begin{multichoice}[randomizeorder, selectmultiple]
\itemtrue ja
\itemfalse nej
\end{multichoice}

\hint
Test

\end{exercise}


\begin{exercise}{Matrix test 1}

Skriv ligningssystemet herunder på matrix form
\[
2x + 3y = 2 \qquad \qquad
x - y = 1
\]

Dvs. skriv det på formen
\[
A \cdot \vec{x} = \vec{b} \qquad \qquad \vec{x} = (x, y)^T
\]
og angiv værdierne i \(A\) og \(\vec{b}\).

\( A = \)
\begin{answermatrix}
2 & 3 \\
1 & -1
\end{answermatrix}


\( \vec{b} = \)
\begin{answermatrix}
2 \\
1
\end{answermatrix}

\hint
Test
\end{exercise}



\end{document}
