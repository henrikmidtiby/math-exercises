\documentclass{article}
\usepackage[utf8]{inputenc}
\usepackage{todonotes}
\usepackage{graphicx}
\usepackage{amsmath}
\usepackage[colorlinks, linkcolor=blue, citecolor=blue, urlcolor=blue]{hyperref}

\newenvironment{exercise}[1]{\newpage\section{#1}}{}
\newcommand{\answerbox}[1]{\fbox{$#1$}}
\newcommand{\hint}{\subsection*{Hint}}

\begin{document}
\tableofcontents
\newpage

\begin{exercise}{Bestem reel del på polær form 1-1}

Bestem den reelle del af det komplekse tal $z=5 \cdot e^{i \pi}$.

\answerbox{-5}


\hint 

Udnyt at modulus og argumentet kendes


\hint

Benyt at det komplekse tal i et argand diagram danner en retvinklet trekant med den reelle akse. 

\hint 

Vha. cosinus kan vi bestemme længden af den hosliggende katete(den reelle del).
\[
Re(z) =  5 \cdot \cos(\pi) = 5 \cdot (-1) = -5 
\]

\end{exercise}

\newpage

\begin{exercise}{Bestem reel del på polær form 1-2}
	
	Bestem den reelle del af det komplekse tal $z=2 \cdot e^{i \frac{\pi}{2}}$.
	
	\answerbox{0}
	
	
	\hint 
	
	Udnyt at modulus og argumentet kendes
	
	
	\hint
	
	Benyt at det komplekse tal i et argand diagram danner en retvinklet trekant med den reelle akse. 
	
	\hint 
	
	Vha. cosinus kan vi bestemme længden af den hosliggende katete(den reelle del).
	\[
	Re(z) =  2 \cdot \cos\left(\frac{\pi}{2}\right) = 2 \cdot 0 = 0
	\]
	
\end{exercise}

\newpage

\begin{exercise}{Bestem reel del på polær form 1-3}
	
	Bestem den reelle del af det komplekse tal $z=4 \cdot e^{i 2\pi}$.
	
	\answerbox{4}
	
	
	\hint 
	
	Udnyt at modulus og argumentet kendes
	
	
	\hint
	
	Benyt at det komplekse tal i et argand diagram danner en retvinklet trekant med den reelle akse. 
	
	\hint 
	
	Vha. cosinus kan vi bestemme længden af den hosliggende katete(den reelle del).
	\[
	Re(z) =  4 \cdot \cos(2\pi) = 4 \cdot 1 = 4
	\]
	
\end{exercise}

\newpage

\begin{exercise}{Bestem reel del på polær form 1-4}
	
	Bestem den reelle del af det komplekse tal $z=6 \cdot e^{i \frac{2 \pi}{3}}$.
	
	\answerbox{-3}
	
	
	\hint 
	
	Udnyt at modulus og argumentet kendes
	
	
	\hint
	
	Benyt at det komplekse tal i et argand diagram danner en retvinklet trekant med den reelle akse. 
	
	\hint 
	
	Vha. cosinus kan vi bestemme længden af den hosliggende katete(den reelle del).
	\[
	Re(z) =  6 \cdot \cos\left(\frac{2 \pi}{3} \right) = 6 \cdot (-0.5) = -3
	\]
	
\end{exercise}

\newpage

\begin{exercise}{Bestem reel del på polær form 1-5}
	
	
Bestem den reelle del af det komplekse tal $z=8 \cdot e^{i \frac{\pi}{3}}$.

\answerbox{4}


\hint 

Udnyt at modulus og argumentet kendes


\hint

Benyt at det komplekse tal i et argand diagram danner en retvinklet trekant med den reelle akse. 

\hint 

Vha. cosinus kan vi bestemme længden af den hosliggende katete(den reelle del).
\[
Re(z) =  8 \cdot \cos\left(\frac{\pi}{3} \right) = 8 \cdot 0.5 = 4
\]
	
\end{exercise}

\newpage

\begin{exercise}{Bestem reel del på polær form 1-6}
	
	
Bestem den reelle del af det komplekse tal $z=10 \cdot e^{i \frac{3 \pi}{2}}$.

\answerbox{0}


\hint 

Udnyt at modulus og argumentet kendes


\hint

Benyt at det komplekse tal i et argand diagram danner en retvinklet trekant med den reelle akse. 

\hint 

Vha. cosinus kan vi bestemme længden af den hosliggende katete(den reelle del).
\[
Re(z) =  10 \cdot \cos\left(\frac{3 \pi}{2} \right) = 10 \cdot 0 = 0
\]
	
\end{exercise}

\newpage

\begin{exercise}{Bestem reel del på polær form 1-7}
	
	
Bestem den reelle del af det komplekse tal $z=4 \cdot e^{i \frac{4 \pi}{3}}$.

\answerbox{-2}


\hint 

Udnyt at modulus og argumentet kendes


\hint

Benyt at det komplekse tal i et argand diagram danner en retvinklet trekant med den reelle akse. 

\hint 

Vha. cosinus kan vi bestemme længden af den hosliggende katete(den reelle del).
\[
Re(z) =  4 \cdot \cos\left(\frac{4 \pi}{3} \right) = 4 \cdot (-0.5) = -2
\]
	
\end{exercise}

\newpage

\begin{exercise}{Bestem reel del på polær form 1-8}
	
	
Bestem den reelle del af det komplekse tal $z=5 \cdot e^{i \frac{5 \pi}{3}}$.

\answerbox{2.5}


\hint 

Udnyt at modulus og argumentet kendes


\hint

Benyt at det komplekse tal i et argand diagram danner en retvinklet trekant med den reelle akse. 

\hint 

Vha. cosinus kan vi bestemme længden af den hosliggende katete(den reelle del).
\[
Re(z) =  5 \cdot \cos\left(\frac{5 \pi}{3} \right) = 5 \cdot 0.5 = 2.5
\]
	
\end{exercise}

\end{document}