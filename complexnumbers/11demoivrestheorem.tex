\documentclass{article}
\usepackage[utf8]{inputenc}
\usepackage{todonotes}
\usepackage{graphicx}
\usepackage{amsmath}
\usepackage[colorlinks, linkcolor=blue, citecolor=blue, urlcolor=blue]{hyperref}

\newenvironment{exercise}[1]{\newpage\section{#1}}{}
\newcommand{\answerbox}[1]{\fbox{$#1$}}
\newcommand{\hint}{\subsection*{Hint}}

\begin{document}
\tableofcontents
\newpage

\begin{exercise}{De Moivres formel 1-1}

Find alle løsninger til $w^4=-16$. \\
Arranger dem efter stigende argumenter.

$w_1$ = \answerbox{2 \cdot e^{i \frac{\pi}{4}}}		$w_2$ = \answerbox{2 \cdot e^{i \frac{3 \pi}{4}}}		$w_3$ = \answerbox{2 \cdot e^{i \frac{5 \pi}{4}}}		$w_4$ = \answerbox{2 \cdot e^{i \frac{7 \pi}{4}}}


\hint 

Skriv højresiden på polær form
\[
w^4 = 16 \cdot e^{i \pi}
\]


\hint

Skriv $w$ på polær form $w = r_w \cdot e^{i \theta_w}$
\[
\left(r_w \cdot e^{i \theta_w}\right)^4 = r_w^4 \cdot e^{i \theta_w \cdot 4} = 16 \cdot e^{i \pi}
\]

\hint 
Match modulus og fase

\[
r_w^4 = 16 \qquad \wedge \qquad e^{i \theta_w \cdot 4} = e^{i \pi}
\]

\hint

Bestem modulus
\[
r_w  =  \sqrt[4]{16} = 2
\]

\hint

Bestem argumentet

\[
\theta_w \cdot 4 = \pi \qquad \Rightarrow \qquad  \theta_w = \frac{\pi}{4}
\]

\hint

Bemærk at $e^{i \theta} = e^{i \theta + 2 \pi \cdot p}$, hvor $p$ er et helt tal.

\hint

Find en anden løsning ($p=1$)
\[
\theta_w \cdot 4 = \pi  + 2 \pi \cdot p \qquad \Rightarrow \qquad  \theta_w = \frac{3 \pi}{4}
\]

\hint

Find en tredje løsning ($p=2$)
\[
\theta_w \cdot 4 = \pi  + 2 \pi \cdot p \qquad \Rightarrow  \qquad  \theta_w = \frac{5 \pi}{4}
\]

\hint

Find en fjerde løsning ($p=3$). Bemærk at dette er den sidste løsning, idet de efterfølgende løsninger blot vil være gentagelser af de fire første. 
\[
\theta_w \cdot 4 = \pi  + 2 \pi \cdot p \qquad \Rightarrow  \qquad  \theta_w = \frac{7 \pi}{4}
\]

\end{exercise}

\newpage

\begin{exercise}{De Moivres formel 1-2}
	
	Find alle løsninger til $w^2=-49$. \\
	Arranger dem efter stigende argumenter.
	
	$w_1$ = \answerbox{7 \cdot e^{i \frac{\pi}{2}}}		$w_2$ = \answerbox{7 \cdot e^{i \frac{3 \pi}{2}}}		
	
	
	\hint 
	
	Skriv højresiden på polær form
	\[
	w^2 = 49 \cdot e^{i \pi}
	\]
	
	
	\hint
	
	Skriv $w$ på polær form $w = r_w \cdot e^{i \theta_w}$
	\[
	\left(r_w \cdot e^{i \theta_w}\right)^2 = r_w^2 \cdot e^{i \theta_w \cdot 2} = 49 \cdot e^{i \pi}
	\]
	
	\hint 
	Match modulus og fase
	
	\[
	r_w^2 = 49 \qquad \wedge \qquad e^{i \theta_w \cdot 2} = e^{i \pi}
	\]
	
	\hint
	
	Bestem modulus
	\[
	r_w  =  \sqrt{49} = 7
	\]
	
	\hint
	
	Bestem argumentet
	
	\[
	\theta_w \cdot 2 = \pi \qquad \Rightarrow \qquad  \theta_w = \frac{\pi}{2}
	\]
	
	\hint
	
	Bemærk at $e^{i \theta} = e^{i \theta + 2 \pi \cdot p}$, hvor $p$ er et helt tal.
	
	\hint
	
	Find en anden løsning ($p=1$). Bemærk at dette er den sidste løsning, idet de efterfølgende løsninger blot vil være gentagelser af de forrige. 
	\[
	\theta_w \cdot 2 = \pi  + 2 \pi \cdot p \qquad \Rightarrow \qquad  \theta_w = \frac{3 \pi}{2}
	\]

	
\end{exercise}

\newpage

\begin{exercise}{De Moivres formel 1-3}
	
	Find alle løsninger til $w^3=-27$. \\
	Arranger dem efter stigende argumenter.
	
	$w_1$ = \answerbox{3 \cdot e^{i \frac{\pi}{3}}}		$w_2$ = \answerbox{3 \cdot e^{i \pi}}		$w_3$ = \answerbox{3 \cdot e^{i \frac{5 \pi}{3}}}	
	
	
	\hint 
	
	Skriv højresiden på polær form
	\[
	w^3 = 27 \cdot e^{i \pi}
	\]
	
	
	\hint
	
	Skriv $w$ på polær form $w = r_w \cdot e^{i \theta_w}$
	\[
	\left(r_w \cdot e^{i \theta_w}\right)^3 = r_w^3 \cdot e^{i \theta_w \cdot 3} = 27 \cdot e^{i \pi}
	\]
	
	\hint 
	Match modulus og fase
	
	\[
	r_w^3 = 27 \qquad \wedge \qquad e^{i \theta_w \cdot 3} = e^{i \pi}
	\]
	
	\hint
	
	Bestem modulus
	\[
	r_w  =  \sqrt[3]{27} = 3
	\]
	
	\hint
	
	Bestem argumentet
	
	\[
	\theta_w \cdot 3 = \pi \qquad \Rightarrow \qquad  \theta_w = \frac{\pi}{3}
	\]
	
	\hint
	
	Bemærk at $e^{i \theta} = e^{i \theta + 2 \pi \cdot p}$, hvor $p$ er et helt tal.
	
	\hint
	
	Find en anden løsning ($p=1$)
	\[
	\theta_w \cdot 3 = \pi  + 2 \pi \cdot p \qquad \Rightarrow \qquad  \theta_w = \pi
	\]
	
	\hint
	
	Find en tredje løsning ($p=2$). Bemærk at dette en den sidste løsning, idet de efterfølgende løsninger blot vil være gentagelser af de forrige.
	\[
	\theta_w \cdot 3 = \pi  + 2 \pi \cdot p \qquad \Rightarrow  \qquad  \theta_w = \frac{5 \pi}{3}
	\]
	
	
\end{exercise}

\newpage

\begin{exercise}{De Moivres formel 1-4}
	
	Find alle løsninger til $w^4=-81$. \\
	Arranger dem efter stigende argumenter.
	
	$w_1$ = \answerbox{3 \cdot e^{i \frac{\pi}{4}}}		$w_2$ = \answerbox{3 \cdot e^{i \frac{3 \pi}{4}}}		$w_3$ = \answerbox{3 \cdot e^{i \frac{5 \pi}{4}}}		$w_4$ = \answerbox{3 \cdot e^{i \frac{7 \pi}{4}}}	 
	
	
	\hint 
	
	Skriv højresiden på polær form
	\[
	w^4 = 81 \cdot e^{i \pi}
	\]
	
	
	\hint
	
	Skriv $w$ på polær form $w = r_w \cdot e^{i \theta_w}$
	\[
	\left(r_w \cdot e^{i \theta_w}\right)^4 = r_w^4 \cdot e^{i \theta_w \cdot 4} = 81 \cdot e^{i \pi}
	\]
	
	\hint 
	Match modulus og fase
	
	\[
	r_w^4 = 81 \qquad \wedge \qquad e^{i \theta_w \cdot 4} = e^{i \pi}
	\]
	
	\hint
	
	Bestem modulus
	\[
	r_w  =  \sqrt[4]{81} = 3
	\]
	
	\hint
	
	Bestem argumentet
	
	\[
	\theta_w \cdot 4 = \pi \qquad \Rightarrow \qquad  \theta_w = \frac{\pi}{4}
	\]
	
	\hint
	
	Bemærk at $e^{i \theta} = e^{i \theta + 2 \pi \cdot p}$, hvor $p$ er et helt tal.
	
	\hint
	
	Find en anden løsning ($p=1$)
	\[
	\theta_w \cdot 4 = \pi  + 2 \pi \cdot p \qquad \Rightarrow \qquad  \theta_w = \frac{3 \pi}{4}
	\]
	
	\hint
	
	Find en tredje løsning ($p=2$)
	\[
	\theta_w \cdot 4 = \pi  + 2 \pi \cdot p \qquad \Rightarrow  \qquad  \theta_w = \frac{5 \pi}{4}
	\]
	
	\hint
	
	Find en fjerde løsning ($p=3$). Bemærk at dette er den sidste løsning, idet de efterfølgende løsninger blot vil være gentagelser af de fire første. 
	\[
	\theta_w \cdot 4 = \pi  + 2 \pi \cdot p \qquad \Rightarrow  \qquad  \theta_w = \frac{7 \pi}{4}
	\]
	
\end{exercise}

\newpage

\begin{exercise}{De Moivres formel 1-5}
	
	Find alle løsninger til $w^2=-25$. \\
	Arranger dem efter stigende argumenter.
	
	$w_1$ = \answerbox{5 \cdot e^{i \frac{\pi}{2}}}		$w_2$ = \answerbox{5 \cdot e^{i \frac{3 \pi}{2}}}		
	
	
	\hint 
	
	Skriv højresiden på polær form
	\[
	w^2 = 25 \cdot e^{i \pi}
	\]
	
	
	\hint
	
	Skriv $w$ på polær form $w = r_w \cdot e^{i \theta_w}$
	\[
	\left(r_w \cdot e^{i \theta_w}\right)^2 = r_w^2 \cdot e^{i \theta_w \cdot 2} = 25 \cdot e^{i \pi}
	\]
	
	\hint 
	Match modulus og fase
	
	\[
	r_w^2 = 25 \qquad \wedge \qquad e^{i \theta_w \cdot 2} = e^{i \pi}
	\]
	
	\hint
	
	Bestem modulus
	\[
	r_w  =  \sqrt{25} = 5
	\]
	
	\hint
	
	Bestem argumentet
	
	\[
	\theta_w \cdot 2 = \pi \qquad \Rightarrow \qquad  \theta_w = \frac{\pi}{2}
	\]
	
	\hint
	
	Bemærk at $e^{i \theta} = e^{i \theta + 2 \pi \cdot p}$, hvor $p$ er et helt tal.
	
	\hint
	
	Find en anden løsning ($p=1$). Bemærk at dette er den sidste løsning, idet de efterfølgende løsninger blot vil være gentagelser af de forrige.
	\[
	\theta_w \cdot 2 = \pi  + 2 \pi \cdot p \qquad \Rightarrow \qquad  \theta_w = \frac{3 \pi}{2}
	\]
	

	
\end{exercise}

\end{document}
