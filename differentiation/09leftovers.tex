\documentclass{article}
\usepackage{tekvideoexercises}

\begin{document}
\tableofcontents
\newpage

\begin{exercise}{Rumgeometri 1}

Find en enhedsvektor der er vinkelret på det plan
der indholder punkterne $(a, 0, 0)$, $(0, b, 0)$ og
$(0, 0, c)$.
Hvad er arealet af den trekant der er udspændt af 
disse tre punkter?

Intet svar...
\answerbox{0}

\hint
Kan løses ved at finde to vektorer der ligger i planet
og derefter finde en vektor der er vinkelret på 
disse to vektorer.

\hint
Krydsproduktet kan bruges til at finde en vektor der
er vinkelret på to andre vektorer.

\hint
Krydsproduktet kan også bruges til at bestemme arealet.

\end{exercise}


\begin{exercise}{Rumgeometri 2}

Find en enhedsvektor med en positiv $\mathbf{k}$ komponent, 
der er vinkelret på både
$2 \mathbf{i} - \mathbf{j} - 2 \mathbf{k}$
og 
$2 \mathbf{i} - 3\mathbf{j} + \mathbf{k}$.

Intet svar...
\answerbox{0}

\hint
Krydsproduktet kan bruges til at finde en vektor der
er vinkelret på to andre vektorer.

\hint
Gang vektoren med en skalar for at få det rigtige fortegn
på den ene koordinat.

\end{exercise}


\begin{exercise}{Rumgeometri 3}

Bestem ligningen for det plan der går igennem
punktet $(0, 2, -3)$ og er vinkelret på vektoren
$4 \mathbf{i} - \mathbf{j} - 2 \mathbf{k}$.

Intet svar...
\answerbox{0}

\hint
\[
\vec{n} \cdot \left( \vec{p} - \vec{p}_0 \right) = 0
\]

\end{exercise}


\begin{exercise}{Rumgeometri 4}

Bestem ligningen for det plan der går igennem
de tre punter: (1, 1, 0), (2, 0, 2) og (0, 3, 3). 

Intet svar...
\answerbox{0}

\hint
Kan løses ved at finde to vektorer der ligger i planet
og derefter finde en vektor der er vinkelret på 
disse to vektorer.

\hint
Krydsproduktet kan bruges til at finde en vektor der
er vinkelret på to andre vektorer.

\hint
\[
\vec{n} \cdot \left( \vec{p} - \vec{p}_0 \right) = 0
\]


\end{exercise}


\begin{exercise}{Rumgeometri 5}

Under hvilke geometriske forhold vil tre punkter i 
$\mathcal{R}^3$ ikke beskrive et unikt plan der går igennem
alle punkterne?
Hvordan kan denne betingelse skrives matematisk / algebraisk
ud fra positionsvektorerne $\mathbf{r}_1$, $\mathbf{r}_2$ og
$\mathbf{r}_3$ for de tre punkter?

Intet svar...
\answerbox{0}

\hint
Kan punkterne placeret uheldigt?

\hint
Hvad hvis de ligger på linje.

\hint
Lav en matrix med de koordinaterne til de tre punkter
i hver linje.
Hvis determinanten af matricen er 0 ligger punkterne på en linje, 
og planet er derfor ikke bestemt.

\end{exercise}


\end{document}
