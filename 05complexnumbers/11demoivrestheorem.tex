\documentclass{article}
\usepackage[utf8]{inputenc}
\usepackage{todonotes}
\usepackage{graphicx}
\usepackage{amsmath}
\usepackage[colorlinks, linkcolor=blue, citecolor=blue, urlcolor=blue]{hyperref}

\newenvironment{exercise}[1]{\newpage\section{#1}}{}
\newcommand{\answerbox}[1]{\fbox{$#1$}}
\newcommand{\hint}{\subsection*{Hint}}

\begin{document}
\tableofcontents
\newpage

\begin{exercise}{De Moivres formel 1-1}

Find alle løsninger til $w^4=-16$. 

Arranger dem efter stigende argumenter.

$w_1$ = \answerbox{2 \cdot e^{i \frac{\pi}{4}}}		

$w_2$ = \answerbox{2 \cdot e^{i \frac{3 \pi}{4}}}		

$w_3$ = \answerbox{2 \cdot e^{i \frac{5 \pi}{4}}}		

$w_4$ = \answerbox{2 \cdot e^{i \frac{7 \pi}{4}}}


\hint 

Skriv højresiden på polær form
\[
w^4 = 16 \cdot e^{i \pi}
\]


\hint

Skriv $w$ på polær form $w = r_w \cdot e^{i \theta_w}$
\[
\left(r_w \cdot e^{i \theta_w}\right)^4 = r_w^4 \cdot e^{i \theta_w \cdot 4} = 16 \cdot e^{i \pi}
\]

\hint 
Match modulus og fase

\[
r_w^4 = 16 \qquad \wedge \qquad e^{i \theta_w \cdot 4} = e^{i \pi}
\]

\hint

Bestem modulus
\[
r_w  =  \sqrt[4]{16} = 2
\]

\hint

Bestem argumentet

\[
\theta_w \cdot 4 = \pi \qquad \Rightarrow \qquad  \theta_w = \frac{\pi}{4}
\]

\hint

Bemærk at $e^{i \theta} = e^{i \theta + 2 \pi \cdot p}$, hvor $p$ er et helt tal.

\hint

Find en anden løsning ($p=1$)
\[
\theta_w \cdot 4 = \pi  + 2 \pi \cdot p \qquad \Rightarrow \qquad  \theta_w = \frac{3 \pi}{4}
\]

\hint

Find en tredje løsning ($p=2$)
\[
\theta_w \cdot 4 = \pi  + 2 \pi \cdot p \qquad \Rightarrow  \qquad  \theta_w = \frac{5 \pi}{4}
\]

\hint

Find en fjerde løsning ($p=3$). 
Bemærk at dette er den sidste løsning, idet de efterfølgende 
løsninger blot vil være gentagelser af de fire første. 
\[
\theta_w \cdot 4 = \pi  + 2 \pi \cdot p \qquad \Rightarrow  \qquad  \theta_w = \frac{7 \pi}{4}
\]

\end{exercise}

\newpage

\begin{exercise}{De Moivres formel 1-2}

Find alle løsninger til $w^2=-49$. 

Arranger dem efter stigende argumenter.

$w_1$ = \answerbox{7 \cdot e^{i \frac{\pi}{2}}}		

$w_2$ = \answerbox{7 \cdot e^{i \frac{3 \pi}{2}}}		


\hint 

Skriv højresiden på polær form
\[
w^2 = 49 \cdot e^{i \pi}
\]


\hint

Skriv $w$ på polær form $w = r_w \cdot e^{i \theta_w}$
\[
\left(r_w \cdot e^{i \theta_w}\right)^2 = r_w^2 \cdot e^{i \theta_w \cdot 2} = 49 \cdot e^{i \pi}
\]

\hint 
Match modulus og fase

\[
r_w^2 = 49 \qquad \wedge \qquad e^{i \theta_w \cdot 2} = e^{i \pi}
\]

\hint

Bestem modulus
\[
r_w  =  \sqrt{49} = 7
\]

\hint

Bestem argumentet

\[
\theta_w \cdot 2 = \pi \qquad \Rightarrow \qquad  \theta_w = \frac{\pi}{2}
\]

\hint

Bemærk at $e^{i \theta} = e^{i \theta + 2 \pi \cdot p}$, hvor $p$ er et helt tal.

\hint

Find en anden løsning ($p=1$). 
Bemærk at dette er den sidste løsning, idet de efterfølgende 
løsninger blot vil være gentagelser af de forrige. 
\[
\theta_w \cdot 2 = \pi  + 2 \pi \cdot p \qquad \Rightarrow \qquad  \theta_w = \frac{3 \pi}{2}
\]


\end{exercise}

\newpage

\begin{exercise}{De Moivres formel 1-3}

Find alle løsninger til $w^3=-27$. 

Arranger dem efter stigende argumenter.

$w_1$ = \answerbox{3 \cdot e^{i \frac{\pi}{3}}}		

$w_2$ = \answerbox{3 \cdot e^{i \pi}}		

$w_3$ = \answerbox{3 \cdot e^{i \frac{5 \pi}{3}}}	


\hint 

Skriv højresiden på polær form
\[
w^3 = 27 \cdot e^{i \pi}
\]


\hint

Skriv $w$ på polær form $w = r_w \cdot e^{i \theta_w}$
\[
\left(r_w \cdot e^{i \theta_w}\right)^3 = r_w^3 \cdot e^{i \theta_w \cdot 3} = 27 \cdot e^{i \pi}
\]

\hint 
Match modulus og fase

\[
r_w^3 = 27 \qquad \wedge \qquad e^{i \theta_w \cdot 3} = e^{i \pi}
\]

\hint

Bestem modulus
\[
r_w  =  \sqrt[3]{27} = 3
\]

\hint

Bestem argumentet

\[
\theta_w \cdot 3 = \pi \qquad \Rightarrow \qquad  \theta_w = \frac{\pi}{3}
\]

\hint

Bemærk at $e^{i \theta} = e^{i \theta + 2 \pi \cdot p}$, hvor $p$ er et helt tal.

\hint

Find en anden løsning ($p=1$)
\[
\theta_w \cdot 3 = \pi  + 2 \pi \cdot p \qquad \Rightarrow \qquad  \theta_w = \pi
\]

\hint

Find en tredje løsning ($p=2$). 
Bemærk at dette en den sidste løsning, idet de efterfølgende 
løsninger blot vil være gentagelser af de forrige.
\[
\theta_w \cdot 3 = \pi  + 2 \pi \cdot p \qquad \Rightarrow  \qquad  \theta_w = \frac{5 \pi}{3}
\]


\end{exercise}

\newpage

\begin{exercise}{De Moivres formel 1-4}

Find alle løsninger til $w^4=-81$. 

Arranger dem efter stigende argumenter.

$w_1$ = \answerbox{3 \cdot e^{i \frac{\pi}{4}}}		

$w_2$ = \answerbox{3 \cdot e^{i \frac{3 \pi}{4}}}		

$w_3$ = \answerbox{3 \cdot e^{i \frac{5 \pi}{4}}}		

$w_4$ = \answerbox{3 \cdot e^{i \frac{7 \pi}{4}}}	 


\hint 

Skriv højresiden på polær form
\[
w^4 = 81 \cdot e^{i \pi}
\]


\hint

Skriv $w$ på polær form $w = r_w \cdot e^{i \theta_w}$
\[
\left(r_w \cdot e^{i \theta_w}\right)^4 = r_w^4 \cdot e^{i \theta_w \cdot 4} = 81 \cdot e^{i \pi}
\]

\hint 
Match modulus og fase

\[
r_w^4 = 81 \qquad \wedge \qquad e^{i \theta_w \cdot 4} = e^{i \pi}
\]

\hint

Bestem modulus
\[
r_w  =  \sqrt[4]{81} = 3
\]

\hint

Bestem argumentet

\[
\theta_w \cdot 4 = \pi \qquad \Rightarrow \qquad  \theta_w = \frac{\pi}{4}
\]

\hint

Bemærk at $e^{i \theta} = e^{i \theta + 2 \pi \cdot p}$, hvor $p$ er et helt tal.

\hint

Find en anden løsning ($p=1$)
\[
\theta_w \cdot 4 = \pi  + 2 \pi \cdot p \qquad \Rightarrow \qquad  \theta_w = \frac{3 \pi}{4}
\]

\hint

Find en tredje løsning ($p=2$)
\[
\theta_w \cdot 4 = \pi  + 2 \pi \cdot p \qquad \Rightarrow  \qquad  \theta_w = \frac{5 \pi}{4}
\]

\hint

Find en fjerde løsning ($p=3$). 
Bemærk at dette er den sidste løsning, idet de efterfølgende 
løsninger blot vil være gentagelser af de fire første. 
\[
\theta_w \cdot 4 = \pi  + 2 \pi \cdot p \qquad \Rightarrow  \qquad  \theta_w = \frac{7 \pi}{4}
\]

\end{exercise}

\newpage

\begin{exercise}{De Moivres formel 1-5}

Find alle løsninger til $w^2=-25$. 

Arranger dem efter stigende argumenter.

$w_1$ = \answerbox{5 \cdot e^{i \frac{\pi}{2}}}		

$w_2$ = \answerbox{5 \cdot e^{i \frac{3 \pi}{2}}}		


\hint 

Skriv højresiden på polær form
\[
w^2 = 25 \cdot e^{i \pi}
\]


\hint

Skriv $w$ på polær form $w = r_w \cdot e^{i \theta_w}$
\[
\left(r_w \cdot e^{i \theta_w}\right)^2 = r_w^2 \cdot e^{i \theta_w \cdot 2} = 25 \cdot e^{i \pi}
\]

\hint 
Match modulus og fase

\[
r_w^2 = 25 \qquad \wedge \qquad e^{i \theta_w \cdot 2} = e^{i \pi}
\]

\hint

Bestem modulus
\[
r_w  =  \sqrt{25} = 5
\]

\hint

Bestem argumentet

\[
\theta_w \cdot 2 = \pi \qquad \Rightarrow \qquad  \theta_w = \frac{\pi}{2}
\]

\hint

Bemærk at $e^{i \theta} = e^{i \theta + 2 \pi \cdot p}$, hvor $p$ er et helt tal.

\hint

Find en anden løsning ($p=1$). 
Bemærk at dette er den sidste løsning, idet de efterfølgende 
løsninger blot vil være gentagelser af de forrige.
\[
\theta_w \cdot 2 = \pi  + 2 \pi \cdot p \qquad \Rightarrow \qquad  \theta_w = \frac{3 \pi}{2}
\]



\end{exercise}

\newpage

\begin{exercise}{De Moivres formel 1-6}

Find alle løsninger til $z^4=-1-i$. 

Arranger dem efter stigende argumenter.

$z_1$ = \answerbox{2^{\frac{1}{8}} \cdot e^{i \frac{5\pi}{16}}}		

$z_2$ = \answerbox{2^{\frac{1}{8}} \cdot e^{i \frac{13\pi}{16}}}		

$z_3$ = \answerbox{2^{\frac{1}{8}} \cdot e^{i \frac{21\pi}{16}}}		

$z_4$ = \answerbox{2^{\frac{1}{8}} \cdot e^{i \frac{29\pi}{16}}}


\hint 

Indtegn det komplekse tal i et argand diagram. 

\hint 

Ud fra argand diagrammet kan højresiden skrives på polær form
\[
z^4 = \sqrt{(-1)^2+(-1)^2} \cdot e^{i \frac{5 \pi}{4}} = 2^{\frac{1}{2}} \cdot e^{i \frac{5 \pi}{4}} 
\]


\hint

Skriv $z$ på polær form $z = r_z \cdot e^{i \theta_z}$
\[
\left(r_z \cdot e^{i \theta_z}\right)^4 = r_z^4 \cdot e^{i \theta_z \cdot 4} = 2^{\frac{1}{2}} \cdot e^{i \frac{5 \pi}{4}} 
\]

\hint 
Match modulus og fase

\[
r_z^4 = 2^{\frac{1}{2}} \qquad \wedge \qquad e^{i \theta_z \cdot 4} = e^{i \frac{5 \pi}{4}} 
\]

\hint

Bestem modulus
\[
r_z  =  \left(2^{\frac{1}{2}} \right)^{\frac{1}{4}} = 2^{\frac{1}{8}}
\]

\hint

Bestem argumentet

\[
\theta_z \cdot 4 = \frac{5 \pi}{4} \qquad \Rightarrow \qquad  \theta_z = \frac{5\pi}{16}
\]

\hint

Bemærk at $e^{i \theta} = e^{i \theta + 2 \pi \cdot p}$, hvor $p$ er et helt tal.

\hint

Find en anden løsning ($p=1$)
\[
\theta_z \cdot 4 = \frac{5\pi}{4}  + 2 \pi \cdot p \qquad \Rightarrow \qquad  \theta_z = \frac{13 \pi}{16}
\]

\hint

Find en tredje løsning ($p=2$)
\[
\theta_z \cdot 4 =  \frac{5\pi}{4}  + 2 \pi \cdot p \qquad \Rightarrow  \qquad  \theta_z = \frac{21 \pi}{16}
\]

\hint

Find en fjerde løsning ($p=3$). 
Bemærk at dette er den sidste løsning, idet de efterfølgende 
løsninger blot vil være gentagelser af de fire første. 
\[
\theta_z \cdot 4 = \frac{5\pi}{4} + 2 \pi \cdot p \qquad \Rightarrow  \qquad  \theta_z = \frac{29 \pi}{16}
\]

\end{exercise}

\newpage

\begin{exercise}{De Moivres formel 1-7}

Find alle løsninger til $z^2=-2+2i$. 

Arranger dem efter stigende argumenter.

$z_1$ = \answerbox{8^{\frac{1}{4}} \cdot e^{i \frac{3\pi}{8}}}		

$z_2$ = \answerbox{8^{\frac{1}{4}} \cdot e^{i \frac{11\pi}{8}}}	

\hint 

Indtegn det komplekse tal i et argand diagram. 

\hint 

Ud fra argand diagrammet kan højresiden skrives på polær form
\[
z^2 = \sqrt{(-2)^2+2^2} \cdot e^{i \frac{3 \pi}{4}} = 8^{\frac{1}{2}} \cdot e^{i \frac{3 \pi}{4}} 
\]


\hint

Skriv $z$ på polær form $z = r_z \cdot e^{i \theta_z}$
\[
\left(r_z \cdot e^{i \theta_z}\right)^2 = r_z^2 \cdot e^{i \theta_z \cdot 2} = 8^{\frac{1}{2}} \cdot e^{i \frac{3 \pi}{4}} 
\]

\hint 
Match modulus og fase

\[
r_z^2 = 8^{\frac{1}{2}}  \qquad \wedge \qquad e^{i \theta_z \cdot 2} = e^{i \frac{3 \pi}{4}} 
\]

\hint

Bestem modulus
\[
r_z  =  \left(8^{\frac{1}{2}} \right)^{\frac{1}{2}} = 8^{\frac{1}{4}}
\]

\hint

Bestem argumentet

\[
\theta_z \cdot 2 = \frac{3 \pi}{4} \qquad \Rightarrow \qquad  \theta_z = \frac{3\pi}{8}
\]

\hint

Bemærk at $e^{i \theta} = e^{i \theta + 2 \pi \cdot p}$, hvor $p$ er et helt tal.

\hint

Find en anden løsning ($p=1$). 
Bemærk at dette er den sidste løsning, idet de efterfølgende 
løsninger blot vil være gentagelser af de forrige. 
\[
\theta_z \cdot 2 = \frac{3 \pi}{4}  + 2 \pi \cdot p \qquad \Rightarrow \qquad  \theta_z = \frac{11 \pi}{8}
\]


\end{exercise}

\newpage

\begin{exercise}{De Moivres formel 1-8}

Find alle løsninger til $z^3=i$. 

Arranger dem efter stigende argumenter.

$z_1$ = \answerbox{e^{i \frac{\pi}{6}}}		

$z_2$ = \answerbox{e^{i \frac{5\pi}{6}}}		

$z_3$ = \answerbox{e^{i \frac{3\pi}{2}}}		


\hint 

Indtegn det komplekse tal i et argand diagram. 

\hint 

Ud fra argand diagrammet kan højresiden skrives på polær form
\[
z^3 = \sqrt{1^2} \cdot e^{i \frac{\pi}{2}} = 1 \cdot e^{i \frac{\pi}{2}} 
\]


\hint

Skriv $z$ på polær form $z = r_z \cdot e^{i \theta_z}$
\[
\left(r_z \cdot e^{i \theta_z}\right)^3 = r_z^3 \cdot e^{i \theta_z \cdot 3} = 1 \cdot e^{i \frac{\pi}{2}} 
\]

\hint 
Match modulus og fase

\[
r_z^3 = 1 \qquad \wedge \qquad e^{i \theta_z \cdot 3} = e^{i \frac{\pi}{2}} 
\]

\hint

Bestem modulus
\[
r_z  =  \left(1 \right)^{\frac{1}{3}} = 1
\]

\hint

Bestem argumentet

\[
\theta_z \cdot 3 = \frac{\pi}{2} \qquad \Rightarrow \qquad  \theta_z = \frac{\pi}{6}
\]

\hint

Bemærk at $e^{i \theta} = e^{i \theta + 2 \pi \cdot p}$, hvor $p$ er et helt tal.

\hint

Find en anden løsning ($p=1$)
\[
\theta_z \cdot 3 = \frac{\pi}{2}  + 2 \pi \cdot p \qquad \Rightarrow \qquad  \theta_z = \frac{5 \pi}{6}
\]

\hint

Find en tredje løsning ($p=2$). 
Bemærk at dette er den sidste løsning, idet de efterfølgende 
løsninger blot vil være gentagelser af de forrige.
\[
\theta_z \cdot 3 =  \frac{\pi}{2}  + 2 \pi \cdot p \qquad \Rightarrow  \qquad  \theta_z = \frac{9 \pi}{6} = \frac{3 \pi}{2}
\]


\end{exercise}

\newpage

\begin{exercise}{De Moivres formel 1-9}

Find alle løsninger til $z^4=2-2i$. 

Arranger dem efter stigende argumenter.

$z_1$ = \answerbox{8^{\frac{1}{8}} \cdot e^{i \frac{7\pi}{16}}}		

$z_2$ = \answerbox{8^{\frac{1}{8}} \cdot e^{i \frac{15\pi}{16}}}		

$z_3$ = \answerbox{8^{\frac{1}{8}} \cdot e^{i \frac{23\pi}{16}}}		

$z_4$ = \answerbox{8^{\frac{1}{8}} \cdot e^{i \frac{31\pi}{16}}}


\hint 

Indtegn det komplekse tal i et argand diagram. 

\hint 

Ud fra argand diagrammet kan højresiden skrives på polær form
\[
z^4 = \sqrt{2^2+(-2)^2} \cdot e^{i \frac{7 \pi}{4}} = 8^{\frac{1}{2}} \cdot e^{i \frac{7 \pi}{4}} 
\]


\hint

Skriv $z$ på polær form $z = r_z \cdot e^{i \theta_z}$
\[
\left(r_z \cdot e^{i \theta_z}\right)^4 = r_z^4 \cdot e^{i \theta_z \cdot 4} = 8^{\frac{1}{2}} \cdot e^{i \frac{7 \pi}{4}} 
\]

\hint 
Match modulus og fase

\[
r_z^4 = 8^{\frac{1}{2}} \qquad \wedge \qquad e^{i \theta_z \cdot 4} = e^{i \frac{7 \pi}{4}} 
\]

\hint

Bestem modulus
\[
r_z  =  \left(8^{\frac{1}{2}} \right)^{\frac{1}{4}} = 8^{\frac{1}{8}}
\]

\hint

Bestem argumentet

\[
\theta_z \cdot 4 = \frac{7 \pi}{4} \qquad \Rightarrow \qquad  \theta_z = \frac{7\pi}{16}
\]

\hint

Bemærk at $e^{i \theta} = e^{i \theta + 2 \pi \cdot p}$, hvor $p$ er et helt tal.

\hint

Find en anden løsning ($p=1$)
\[
\theta_z \cdot 4 = \frac{7\pi}{4}  + 2 \pi \cdot p \qquad \Rightarrow \qquad  \theta_z = \frac{15 \pi}{16}
\]

\hint

Find en tredje løsning ($p=2$)
\[
\theta_z \cdot 4 = \frac{7\pi}{4}   + 2 \pi \cdot p \qquad \Rightarrow  \qquad  \theta_z = \frac{23 \pi}{16}
\]

\hint

Find en fjerde løsning ($p=3$). 
Bemærk at dette er den sidste løsning, idet de efterfølgende 
løsninger blot vil være gentagelser af de fire første. 
\[
\theta_z \cdot 4 = \frac{7\pi}{4}  + 2 \pi \cdot p \qquad \Rightarrow  \qquad  \theta_z = \frac{31 \pi}{16}
\]

\end{exercise}

\newpage

\begin{exercise}{De Moivres formel 1-10}

Find alle løsninger til $z^3=1+i$. 

Arranger dem efter stigende argumenter.

$z_1$ = \answerbox{2^{\frac{1}{6}} \cdot e^{i \frac{\pi}{12}}}		

$z_2$ = \answerbox{2^{\frac{1}{6}} \cdot e^{i \frac{3\pi}{4}}}		

$z_3$ = \answerbox{2^{\frac{1}{6}} \cdot e^{i \frac{17\pi}{12}}}	


\hint 

Indtegn det komplekse tal i et argand diagram. 

\hint 

Ud fra argand diagrammet kan højresiden skrives på polær form
\[
z^3 = \sqrt{(1)^2+(1)^2} \cdot e^{i \frac{\pi}{4}} = 2^{\frac{1}{2}} \cdot e^{i \frac{\pi}{4}} 
\]


\hint

Skriv $z$ på polær form $z = r_z \cdot e^{i \theta_z}$
\[
\left(r_z \cdot e^{i \theta_z}\right)^3 = r_z^3 \cdot e^{i \theta_z \cdot 3} = 2^{\frac{1}{2}} \cdot e^{i \frac{ \pi}{4}} 
\]

\hint 
Match modulus og fase

\[
r_z^3 = 2^{\frac{1}{2}} \qquad \wedge \qquad e^{i \theta_z \cdot 3} = e^{i \frac{\pi}{4}} 
\]

\hint

Bestem modulus
\[
r_z  =  \left(2^{\frac{1}{2}} \right)^{\frac{1}{3}} = 2^{\frac{1}{6}}
\]

\hint

Bestem argumentet

\[
\theta_z \cdot 3 = \frac{\pi}{4} \qquad \Rightarrow \qquad  \theta_z = \frac{\pi}{12}
\]

\hint

Bemærk at $e^{i \theta} = e^{i \theta + 2 \pi \cdot p}$, hvor $p$ er et helt tal.

\hint

Find en anden løsning ($p=1$)
\[
\theta_z \cdot 3 = \frac{\pi}{4}  + 2 \pi \cdot p \qquad \Rightarrow \qquad  \theta_z = \frac{9 \pi}{12} = \frac{3\pi}{4}
\]

\hint

Find en tredje løsning ($p=2$). 
Bemærk at dette er den sidste løsning, idet de efterfølgende 
løsninger blot vil være gentagelser af de forrige.
\[
\theta_z \cdot 3 = \frac{\pi}{4}  + 2 \pi \cdot p \qquad \Rightarrow  \qquad  \theta_z = \frac{17 \pi}{12}
\]


\end{exercise}



\end{document}
