\documentclass{article}
\usepackage[utf8]{inputenc}
\usepackage{todonotes}
\usepackage{graphicx}
\usepackage{amsmath}
\usepackage[colorlinks, linkcolor=blue, citecolor=blue, urlcolor=blue]{hyperref}

\newenvironment{exercise}[1]{\newpage\section{#1}}{}
\newcommand{\answerbox}[1]{\fbox{$#1$}}
\newcommand{\hint}{\subsection*{Hint}}

\begin{document}
\tableofcontents
\newpage

\begin{exercise}{Division på polær form 1-1}

Bestem $ \frac{5 \cdot e^{i \pi}}{10 \cdot e^{i \frac{2 \pi}{3}}}$.

\answerbox{\frac{1}{2} \cdot e^{i \frac{\pi}{3}}}


\hint 

Bestem modulus


\hint

Divider de to moduli
\[
\frac{5 \cdot e^{i \pi}}{10 \cdot e^{i \frac{2 \pi}{3}}} = \frac{1 \cdot e^{i \pi}}{2 \cdot e^{i \frac{2 \pi}{3}}}
\]

\hint 

Benyt potensregnereglen
\[
\frac{x^n}{x^m} = x^{n-m}
\]

\hint

Det giver
\[
= \frac{1}{2} \cdot e^{(i \pi - i\frac{2 \pi}{3}) }
\]

\hint

Faktoriser $i$ og bestem fasen
\[
= \frac{1}{2} \cdot e^{i( \pi - \frac{2 \pi}{3})}  = \frac{1}{2} \cdot e^{i \frac{\pi}{3}}
\]


\end{exercise}

\newpage

\begin{exercise}{Division på polær form 1-2}
	
	Bestem $ \frac{4 \cdot e^{i \frac{\pi}{2}}}{2 \cdot e^{i \frac{\pi}{3}}}$.
	
	\answerbox{ 2 \cdot e^{i \frac{\pi}{6}} 	}
	
	
	\hint 
	
	Bestem modulus
	
	
	\hint
	
	Divider de to moduli
	\[
	\frac{4 \cdot e^{i \frac{\pi}{2}}}{2 \cdot e^{i \frac{\pi}{3}}} = \frac{2 \cdot e^{i \frac{\pi}{2}}}{e^{i \frac{\pi}{3}}}
	\]
	
	\hint 
	
	Benyt potensregnereglen
	\[
	\frac{x^n}{x^m} = x^{n-m}
	\]
	
	\hint
	
	Det giver
	\[
	= 2 \cdot e^{(i \frac{\pi}{2}-i \frac{\pi}{3})}
	\]
	
	\hint
	
	Faktoriser $i$ og sæt brøker på fællesnævner
	\[
	= 2 \cdot e^{i( \frac{\pi}{2}- \frac{\pi}{3})}  = 2 \cdot e^{i( \frac{3\pi}{6}- \frac{2\pi}{6})} 	
	\]
	
	\hint
	Bestem fase
	\[
	 = 2 \cdot e^{i \frac{\pi}{6}} 	
	\]
	
\end{exercise}

\newpage

\begin{exercise}{Division på polær form 1-3}
	
	Bestem $ \frac{8 \cdot e^{i \frac{3 \pi}{2}}}{2 \cdot e^{i \frac{\pi}{2}}}$.
	
	\answerbox{4 \cdot e^{i \pi}}
	
	
	\hint 
	
	Bestem modulus
	
	
	\hint
	
	Divider de to moduli
	\[
	\frac{8 \cdot e^{i \frac{3 \pi}{2}}}{2 \cdot e^{i \frac{\pi}{2}}} = \frac{4 \cdot e^{i \frac{3 \pi}{2}}}{e^{i \frac{\pi}{2}}}
	\]
	
	\hint 
	
	Benyt potensregnereglen
	\[
	\frac{x^n}{x^m} = x^{n-m}
	\]
	
	\hint
	
	Det giver
	\[
	= 4 \cdot e^{(i \frac{3 \pi}{2}-i \frac{\pi}{2})}
	\]
	
	\hint
	
	Faktoriser $i$ og bestem fasen
	\[
	= 4 \cdot e^{i( \frac{3 \pi}{2}-\frac{\pi}{2})} = 4 \cdot e^{i \pi}
	\]
	
	
\end{exercise}

\newpage

\begin{exercise}{Division på polær form 1-4}
	
	Bestem $ \frac{20\cdot e^{i \frac{\pi}{3}}}{5 \cdot e^{i \frac{\pi}{4}}}$.
	
	\answerbox{4 \cdot e^{i \frac{\pi}{12}}}
	
	
	\hint 
	
	Bestem modulus
	
	
	\hint
	
	Divider de to moduli
	\[
	\frac{20\cdot e^{i \frac{\pi}{3}}}{5 \cdot e^{i \frac{\pi}{4}}} = \frac{4\cdot e^{i \frac{\pi}{3}}}{ e^{i \frac{\pi}{4}}}
	\]
	
	\hint 
	
	Benyt potensregnereglen
	\[
	\frac{x^n}{x^m} = x^{n-m}
	\]
	
	\hint
	
	Det giver
	\[
	= 4\cdot e^{(i \frac{\pi}{3}-i \frac{\pi}{4})}
	\]
	
	\hint
	
	Faktoriser $i$ og sæt brøker på fællesnævner
	\[
	= 4\cdot e^{i( \frac{\pi}{3}- \frac{\pi}{4})} = 4\cdot e^{i( \frac{4\pi}{12}- \frac{3\pi}{12})} 
	\]
	
	\hint
	
	Bestem fase
	\[
	= 4 \cdot e^{i \frac{\pi}{12}}
	\]
	
	
\end{exercise}

\newpage

\begin{exercise}{Division på polær form 1-5}
	
	Bestem $ \frac{3 \cdot e^{i \frac{\pi}{5}}}{9 \cdot e^{i \frac{\pi}{20}}}$.
	
	\answerbox{\frac{1}{3} \cdot e^{ i \frac{3\pi}{20} }}
	
	
	\hint 
	
	Bestem modulus
	
	
	\hint
	
	Divider de to moduli
	\[
	\frac{3 \cdot e^{i \frac{\pi}{5}}}{9 \cdot e^{i \frac{\pi}{20}}} = \frac{1 \cdot e^{i \frac{\pi}{5}}}{3 \cdot e^{i \frac{\pi}{20}}} 
	\]
	
	\hint 
	
	Benyt potensregnereglen
	\[
	\frac{x^n}{x^m} = x^{n-m}
	\]
	
	\hint
	
	Det giver
	\[
	=\frac{1}{3} \cdot e^{ (i \frac{\pi}{5}-i \frac{\pi}{20}) }
	\]
	
	\hint
	
	Faktoriser $i$ og  sæt brøker på fællesnævner
	\[
	= \frac{1}{3} \cdot e^{ i( \frac{\pi}{5}-\frac{\pi}{20}) }  =	\frac{1}{3} \cdot e^{ i( \frac{4\pi}{20}-\frac{\pi}{20}) }
	\]
	
	\hint
	
	Bestem fasen
	\[
	= \frac{1}{3} \cdot e^{ i \frac{3\pi}{20} }
	\]
	
\end{exercise}

\newpage

\begin{exercise}{Division på polær form 1-6}
	
	Bestem $ \frac{3 \cdot e^{i \frac{3\pi}{2}}}{12 \cdot e^{i \frac{\pi}{3}}}$.
	
	\answerbox{\frac{1}{4} \cdot e^{i \frac{7\pi}{6}}}
	
	
	\hint 
	
	Bestem modulus
	
	
	\hint
	
	Divider de to moduli
	\[
	\frac{3 \cdot e^{i \frac{3\pi}{2}}}{12 \cdot e^{i \frac{\pi}{3}}} = \frac{1 \cdot e^{i \frac{3\pi}{2}}}{4 \cdot e^{i \frac{\pi}{3}}}
	\]
	
	\hint 
	
	Benyt potensregnereglen
	\[
	\frac{x^n}{x^m} = x^{n-m}
	\]
	
	\hint
	
	Det giver
	\[
	= \frac{1}{4} \cdot e^{(i \frac{3 \pi}{2}-i \frac{\pi}{3})}
	\]
	
	\hint
	
	Faktoriser $i$ og sæt på brøker på fællesnævner
	\[
	= \frac{1}{4} \cdot e^{i( \frac{3 \pi}{2}- \frac{\pi}{3})} = \frac{1}{4} \cdot e^{i( \frac{9 \pi}{6}- \frac{2\pi}{6})} 
	\]
	
	\hint
	
	Bestem fasen
	\[
	= \frac{1}{4} \cdot e^{i \frac{7\pi}{6}}
	\]
	
	
\end{exercise}

\newpage

\begin{exercise}{Division på polær form 1-7}
	
	Bestem $ \frac{3 \cdot e^{i 5\pi}}{3 \cdot e^{i 4\pi}}$.
	
	\answerbox{e^{i \pi}}
	
	
	\hint 
	
	Bestem modulus
	
	
	\hint
	
	Divider de to moduli
	\[
	\frac{3 \cdot e^{i 5\pi}}{3 \cdot e^{i 4\pi}} = \frac{e^{i 5\pi}}{e^{i 4\pi}}
	\]
	
	\hint 
	
	Benyt potensregnereglen
	\[
	\frac{x^n}{x^m} = x^{n-m}
	\]
	
	\hint
	
	Det giver
	\[
	= e^{(i 5\pi - i 4\pi)}	
	\]
	
	\hint
	
	Faktoriser $i$ og bestem fasen
	\[
	=  e^{i(5\pi - 4\pi)} = e^{i \pi}
	\]
	
	
\end{exercise}

\newpage

\begin{exercise}{Division på polær form 1-8}
	
	Bestem $ \frac{27\cdot e^{i 3 \pi}}{9 \cdot e^{i \frac{3 \pi}{2}}}$.
	
	\answerbox{3 \cdot e^{i \frac{3\pi}{2}}}
	
	
	\hint 
	
	Bestem modulus
	
	
	\hint
	
	Divider de to moduli
	\[
	\frac{27\cdot e^{i 3 \pi}}{9 \cdot e^{i \frac{3 \pi}{2}}} = \frac{3\cdot e^{i 3 \pi}}{ e^{i \frac{3 \pi}{2}}}
	\]
	
	\hint 
	
	Benyt potensregnereglen
	\[
	\frac{x^n}{x^m} = x^{n-m}
	\]
	
	\hint
	
	Det giver
	\[
	=3\cdot e^{(i 3 \pi- i \frac{3 \pi}{2})} 
	\]
	
	\hint
	
	Faktoriser $i$ og sæt brøker på fællesnævner
	\[
	= 3\cdot e^{i( 3 \pi- \frac{3 \pi}{2})}  = 3 \cdot e^{i( \frac{6 \pi}{2}- \frac{3 \pi}{2})} 
	\]
	
	\hint
	
	Bestem fasen
	\[
	= 3 \cdot e^{i \frac{3\pi}{2}}
	\]
	
\end{exercise}

\end{document}