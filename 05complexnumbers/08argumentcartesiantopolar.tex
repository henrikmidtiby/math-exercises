\documentclass{article}
\usepackage[utf8]{inputenc}
\usepackage{todonotes}
\usepackage{graphicx}
\usepackage{amsmath}
\usepackage[colorlinks, linkcolor=blue, citecolor=blue, urlcolor=blue]{hyperref}

\newenvironment{exercise}[1]{\newpage\section{#1}}{}
\newcommand{\answerbox}[1]{\fbox{$#1$}}
\newcommand{\hint}{\subsection*{Hint}}

\begin{document}
\tableofcontents
\newpage

\begin{exercise}{Bestem argument på kartesisk form 1-1}

Bestem argumentet $\theta$ af det komplekse tal $z=4+4i$, dvs. vinklen imellem det komplekse tal og den reelle akse.

\answerbox{\frac{\pi}{4}}


\hint 

Indtegn det komplekse tal i et argand diagram.


\hint

Bestem argumentet $\theta$ vha. tangens
\[
\tan(\theta)=\frac{\textrm{modstående katete}}{\textrm{hosliggende katete}}
\]

\hint 

Det giver
\[
\theta = \arctan \left(\frac{4}{4}\right) = \frac{\pi}{4}
\]

\hint

Tjek at vinklen er korrekt ud fra argand diagrammet. Læg den nødvendige vinkel oveni for at opnå den korrekte vinkel. 


\end{exercise}

\newpage

\begin{exercise}{Bestem argument på kartesisk form 1-2}
	
	Bestem argumentet $\theta$ af det komplekse tal $z=-5+5i$, dvs. vinklen imellem det komplekse tal og den reelle akse.
	
	\answerbox{\frac{3}{4} \pi}
	
	
	\hint 
	
	Indtegn det komplekse tal i et argand diagram.
	
	
	\hint
	
	Bestem argumentet $\theta$ vha. tangens
	\[
	\tan(\theta)=\frac{\textrm{modstående katete}}{\textrm{hosliggende katete}}
	\]
	
	\hint 
	
	Det giver
	\[
	\theta = \arctan \left(\frac{5}{-5}\right) = \frac{-\pi}{4}
	\]
	
	\hint
	
	Tjek at vinklen er korrekt ud fra argand diagrammet. Læg den nødvendige vinkel oveni for at opnå den korrekte vinkel. 
	
	\[
	\frac{-\pi}{4} + \pi = \frac{3}{4} \pi
	\]
	
\end{exercise}

\newpage

\begin{exercise}{Bestem argument på kartesisk form 1-3}
	
	Bestem argumentet $\theta$ af det komplekse tal $z=5i$, dvs. vinklen imellem det komplekse tal og den reelle akse.
	
	\answerbox{ \frac{\pi}{2}}
	
	
	\hint 
	
	Indtegn det komplekse tal i et argand diagram.
	
	
	\hint
	
	Bestem argumentet $\theta$ vha. tangens
	\[
	\tan(\theta)=\frac{\textrm{modstående katete}}{\textrm{hosliggende katete}}
	\]
	
	\hint 
	
	Det giver
	\[
	\theta = \arctan \left(\frac{5}{0}\right) = \frac{\pi}{2}
	\]
	
	\hint
	
	Tjek at vinklen er korrekt ud fra argand diagrammet. Læg den nødvendige vinkel oveni for at opnå den korrekte vinkel. 
	
	
\end{exercise}

\newpage

\begin{exercise}{Bestem argument på kartesisk form 1-4}
	
	Bestem argumentet $\theta$ af det komplekse tal $z=3$, dvs. vinklen imellem det komplekse tal og den reelle akse.
	
	\answerbox{0}
	
	
	\hint 
	
	Indtegn det komplekse tal i et argand diagram.
	
	
	\hint
	
	Bestem argumentet $\theta$ vha. tangens
	\[
	\tan(\theta)=\frac{\textrm{modstående katete}}{\textrm{hosliggende katete}}
	\]
	
	\hint 
	
	Det giver
	\[
	\theta = \arctan \left(\frac{0}{3}\right) = 0
	\]
	
	\hint
	
	Tjek at vinklen er korrekt ud fra argand diagrammet. Læg den nødvendige vinkel oveni for at opnå den korrekte vinkel. 
	
	
\end{exercise}

\newpage

\begin{exercise}{Bestem argument på kartesisk form 1-5}
	
	Bestem argumentet $\theta$ af det komplekse tal $z=2-2i$, dvs. vinklen imellem det komplekse tal og den reelle akse.
	
	\answerbox{ \frac{7}{4} \pi}
	
	
	\hint 
	
	Indtegn det komplekse tal i et argand diagram.
	
	
	\hint
	
	Bestem argumentet $\theta$ vha. tangens
	\[
	\tan(\theta)=\frac{\textrm{modstående katete}}{\textrm{hosliggende katete}}
	\]
	
	\hint 
	
	Det giver
	\[
	\theta = \arctan \left(\frac{-2}{2}\right) =- \frac{\pi}{4}
	\]
	
	\hint
	
	Tjek at vinklen er korrekt ud fra argand diagrammet. Læg den nødvendige vinkel oveni for at opnå den korrekte vinkel. 
	\[
	- \frac{\pi}{4} + 2 \pi = \frac{7}{4} \pi
	\]
	
\end{exercise}

\newpage

\begin{exercise}{Bestem argument på kartesisk form 1-6}
	
	Bestem argumentet $\theta$ af det komplekse tal $z=-5i$, dvs. vinklen imellem det komplekse tal og den reelle akse.
	
	\answerbox{\frac{3}{2} \pi }
	
	
	\hint 
	
	Indtegn det komplekse tal i et argand diagram.
	
	
	\hint
	
	Bestem argumentet $\theta$ vha. tangens
	\[
	\tan(\theta)=\frac{\textrm{modstående katete}}{\textrm{hosliggende katete}}
	\]
	
	\hint 
	
	Det giver
	\[
	\theta = \arctan \left(\frac{-5}{0}\right) = - \frac{\pi}{2}
	\]
	
	\hint
	
	Tjek at vinklen er korrekt ud fra argand diagrammet. Læg den nødvendige vinkel oveni for at opnå den korrekte vinkel. 
	\[
	- \frac{\pi}{2} + 2 \pi  = \frac{3}{2} \pi  	
	\]
	
	
\end{exercise}

\newpage

\begin{exercise}{Bestem argument på kartesisk form 1-7}
	
	Bestem argumentet $\theta$ af det komplekse tal $z=-2$, dvs. vinklen imellem det komplekse tal og den reelle akse.
	
	\answerbox{\pi}
	
	
	\hint 
	
	Indtegn det komplekse tal i et argand diagram.
	
	
	\hint
	
	Bestem argumentet $\theta$ vha. tangens
	\[
	\tan(\theta)=\frac{\textrm{modstående katete}}{\textrm{hosliggende katete}}
	\]
	
	\hint 
	
	Det giver
	\[
	\theta = \arctan \left(\frac{0}{-2}\right) = 0
	\]
	
	\hint
	
	Tjek at vinklen er korrekt ud fra argand diagrammet. Læg den nødvendige vinkel oveni for at opnå den korrekte vinkel. 
	\[
	0+ \pi = \pi
	\]
	
	
\end{exercise}

\newpage

\begin{exercise}{Bestem argument på kartesisk form 1-8}
	
	Bestem argumentet $\theta$ af det komplekse tal $z=-3-3i$, dvs. vinklen imellem det komplekse tal og den reelle akse.
	
	\answerbox{\frac{5}{4} \pi}
	
	
	\hint 
	
	Indtegn det komplekse tal i et argand diagram.
	
	
	\hint
	
	Bestem argumentet $\theta$ vha. tangens
	\[
	\tan(\theta)=\frac{\textrm{modstående katete}}{\textrm{hosliggende katete}}
	\]
	
	\hint 
	
	Det giver
	\[
	\theta = \arctan \left(\frac{-3}{-3}\right) = \frac{\pi}{4}
	\]
	
	\hint
	
	Tjek at vinklen er korrekt ud fra argand diagrammet. Læg den nødvendige vinkel oveni for at opnå den korrekte vinkel. 
	\[
	\frac{\pi}{4} + \pi = \frac{5}{4} \pi
	\]
	
	
\end{exercise}

\end{document}
