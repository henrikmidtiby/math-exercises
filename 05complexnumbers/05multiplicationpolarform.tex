\documentclass{article}
\usepackage[utf8]{inputenc}
\usepackage{todonotes}
\usepackage{graphicx}
\usepackage{amsmath}
\usepackage[colorlinks, linkcolor=blue, citecolor=blue, urlcolor=blue]{hyperref}

\newenvironment{exercise}[1]{\newpage\section{#1}}{}
\newcommand{\answerbox}[1]{\fbox{$#1$}}
\newcommand{\hint}{\subsection*{Hint}}

\begin{document}
Multiplikation af komplekse tal på polær form
\tableofcontents
\newpage

\begin{exercise}{Multiplikation på polær form 1-1}

Bestem $5 \cdot e^{i \pi} \cdot 10 \cdot e^{i \frac{2 \pi}{3}}$.

\answerbox{50 \cdot e^{i \frac{5\pi }{3}}}


\hint 

Bestem modulus


\hint

Multiplicer de to moduli 
\[
5 \cdot e^{i \pi} \cdot 10 \cdot e^{i \frac{2 \pi}{3}} = 50 \cdot e^{i \pi} \cdot e^{i \frac{2 \pi}{3}}
\]

\hint 

Benyt potensregnereglen
\[
x^n \cdot x^m = x^{n+m}
\]

\hint

Det giver
\[
=50 \cdot e^{(i \pi +i \frac{2 \pi}{3})} 
\]

\hint

Faktoriser $i$ og bestem fasen
\[
= 50 \cdot e^{i( \pi +\frac{2 \pi}{3})}  = 50 \cdot e^{i \frac{5\pi }{3}}
\]


\end{exercise}

\newpage

\begin{exercise}{Multiplikation på polær form 1-2}
	
	Bestem $2 \cdot e^{i \frac{\pi}{2}} \cdot 3 \cdot e^{i \frac{\pi}{2}}$.
	
	\answerbox{6 \cdot e^{i \pi}}
	
	
	\hint 
	
	Bestem modulus
	
	
	\hint
	
	Multiplicer de to moduli 
	\[
	2 \cdot e^{i \frac{\pi}{2}} \cdot 3 \cdot e^{i \frac{\pi}{2}} = 6 e^{i \frac{\pi}{2}} \cdot e^{i \frac{\pi}{2}}
	\]
	
	\hint 
	
	Benyt potensregnereglen
	\[
	x^n \cdot x^m = x^{n+m}
	\]
	
	\hint
	
	Det giver
	\[
	= 6 \cdot e^{(i \frac{\pi}{2}+ i \frac{\pi}{2})}
	\]
	
	\hint
	
	Faktoriser $i$ og bestem fasen
	\[
	= 6 \cdot e^{i( \frac{\pi}{2}+\frac{\pi}{2})} = 6 \cdot e^{i \pi}
	\]
	
	
\end{exercise}


\newpage

\begin{exercise}{Multiplikation på polær form 1-3}
	
	Bestem $4 \cdot e^{i \frac{\pi}{20}} \cdot 5 \cdot e^{i \frac{\pi}{5}}$.
	
	\answerbox{20 \cdot e^{i \frac{\pi}{4}}}
	
	
	\hint 
	
	Bestem modulus
	
	
	\hint
	
	Multiplicer de to moduli 
	\[
	4 \cdot e^{i \frac{\pi}{20}} \cdot 5 \cdot e^{i \frac{\pi}{5}} = 20 \cdot e^{i \frac{\pi}{20}} \cdot e^{i \frac{\pi}{5}}
	\]
	
	\hint 
	
	Benyt potensregnereglen
	\[
	x^n \cdot x^m = x^{n+m}
	\]
	
	\hint
	
	Det giver
	\[
	=20 \cdot e^{(i \frac{\pi}{20}i \frac{\pi}{5})}
	\]
	
	\hint
	
	Faktoriser $i$ og sæt de to brøker på fællesnævner
	\[
	= 20 \cdot e^{i( \frac{\pi}{20} + \frac{\pi}{5})} = 20 \cdot e^{i( \frac{\pi}{20} + \frac{4 \pi}{20})} = 20 \cdot e^{i( \frac{5\pi}{20})}  
	\]
	
	\hint
	
	Reducer udtryk og bestem fasen
	\[
	= 20 \cdot e^{i \frac{\pi}{4}}
	\]
	
\end{exercise}


\newpage

\begin{exercise}{Multiplikation på polær form 1-4}
	
	Bestem $7 \cdot e^{i \pi} \cdot 8 \cdot e^{i \frac{\pi}{2}}$.
	
	\answerbox{56 \cdot e^{i \frac{3\pi}{2}}}
	
	
	\hint 
	
	Bestem modulus
	
	
	\hint
	
	Multiplicer de to moduli 
	\[
	7 \cdot e^{i \pi} \cdot 8 \cdot e^{i \frac{\pi}{2}} = 56 \cdot e^{i \pi} \cdot e^{i \frac{\pi}{2}}
	\]
	
	\hint 
	
	Benyt potensregnereglen
	\[
	x^n \cdot x^m = x^{n+m}
	\]
	
	\hint
	
	Det giver
	\[
	= 56 \cdot e^{(i \pi +i \frac{\pi}{2})}
	\]
	
	\hint
	
	Faktoriser $i$ og bestem fasen
	\[
	= 56 \cdot e^{i( \pi +\frac{\pi}{2})} = 56 \cdot e^{i \frac{3\pi}{2}}
	\]
	
	
\end{exercise}


\newpage

\begin{exercise}{Multiplikation på polær form 1-5}
	
	Bestem $2 \cdot e^{i \frac{\pi}{4}} \cdot 2 \cdot e^{i \frac{\pi}{4}}$.
	
	\answerbox{4 \cdot e^{\frac{\pi}{2}}}
	
	
	\hint 
	
	Bestem modulus
	
	
	\hint
	
	Multiplicer de to moduli 
	\[
	2 \cdot e^{i \frac{\pi}{4}} \cdot 2 \cdot e^{i \frac{\pi}{4}} = 4 \cdot e^{i \frac{\pi}{4}} \cdot e^{i \frac{\pi}{4}}
	\]
	
	\hint 
	
	Benyt potensregnereglen
	\[
	x^n \cdot x^m = x^{n+m}
	\]
	
	\hint
	
	Det giver
	\[
	=4 \cdot e^{(i \frac{\pi}{4} + i \frac{\pi}{4})}
	\]
	
	\hint
	
	Faktoriser $i$ og bestem fasen
	\[
	= 4 \cdot e^{i( \frac{\pi}{4} +\frac{\pi}{4})} = 4 \cdot e^{\frac{\pi}{2}}
	\]
	
	
\end{exercise}


\newpage

\begin{exercise}{Multiplikation på polær form 1-6}
	
	Bestem $3 \cdot e^{i \frac{\pi}{3}} \cdot 3 \cdot e^{i \frac{\pi}{4}}$.
	
	\answerbox{9 \cdot e^{i\frac{7 \pi}{12}}}
	
	
	\hint 
	
	Bestem modulus
	
	
	\hint
	
	Multiplicer de to moduli 
	\[
	3 \cdot e^{i \frac{\pi}{3}} \cdot 3 \cdot e^{i \frac{\pi}{4}} = 9 \cdot e^{i \frac{\pi}{3}}  \cdot e^{i \frac{\pi}{4}}
	\]
	
	\hint 
	
	Benyt potensregnereglen
	\[
	x^n \cdot x^m = x^{n+m}
	\]
	
	\hint
	
	Det giver
	\[
	=9 \cdot e^{(i \frac{\pi}{3}+i \frac{\pi}{4})}
	\]
	
	\hint
	
	Faktoriser $i$ og sæt brøker på fællesnævner
	\[
	= 9 \cdot e^{i( \frac{\pi}{3}+ \frac{\pi}{4})} = 9 \cdot e^{i( \frac{4 \pi}{12} + \frac{3\pi}{12})} 
	\]
	
	\hint
	
	Bestem fase
	\[
	= 9 \cdot e^{i\frac{7 \pi}{12}}
	\]
	
	
\end{exercise}


\newpage

\begin{exercise}{Multiplikation på polær form 1-7}
	
	Bestem $5 \cdot e^{i \frac{\pi}{3}} \cdot 5 \cdot e^{i \frac{\pi}{6}}$.
	
	\answerbox{25 \cdot e^{i \frac{\pi}{2} }}
	
	
	\hint 
	
	Bestem modulus
	
	
	\hint
	
	Multiplicer de to moduli 
	\[
	5 \cdot e^{i \frac{\pi}{3}} \cdot 5 \cdot e^{i \frac{\pi}{6}} = 25 \cdot e^{i \frac{\pi}{3}} \cdot e^{i \frac{\pi}{6}}
	\]
	
	\hint 
	
	Benyt potensregnereglen
	\[
	x^n \cdot x^m = x^{n+m}
	\]
	
	\hint
	
	Det giver
	\[
	=25 \cdot e^{(i \frac{\pi}{3} + i \frac{\pi}{6})}
	\]
	
	\hint
	
	Faktoriser $i$ og sæt brøker på fællesnævner
	\[
	=25 \cdot e^{i( \frac{\pi}{3} + \frac{\pi}{6})} = 25 \cdot e^{i( \frac{2\pi}{6}+ \frac{\pi}{6})}
	\]
	
	\hint
	
	Bestem fase og reducer udtryk
	\[
	= 25 \cdot e^{i \frac{3\pi}{6} } = 25 \cdot e^{i \frac{\pi}{2} }
	\]
	
	
\end{exercise}


\newpage

\begin{exercise}{Multiplikation på polær form 1-8}
	
	Bestem $4 \cdot e^{i \pi} \cdot 8 \cdot e^{i \frac{\pi}{5}}$.
	
	\answerbox{32 \cdot e^{i \frac{6 \pi}{5}}}
	
	
	\hint 
	
	Bestem modulus
	
	
	\hint
	
	Multiplicer de to moduli 
	\[
	4 \cdot e^{i \pi} \cdot 8 \cdot e^{i \frac{\pi}{5}} = 32 \cdot e^{i \pi} \cdot e^{i \frac{\pi}{5}}
	\]
	
	\hint 
	
	Benyt potensregnereglen
	\[
	x^n \cdot x^m = x^{n+m}
	\]
	
	\hint
	
	Det giver
	\[
	=32 \cdot e^{(i \pi + i \frac{\pi}{5})}
	\]
	
	\hint
	
	Faktoriser $i$ og sæt brøker på fællesnævner
	\[
	= 32 \cdot e^{i( \pi+ \frac{\pi}{5})} = 32 \cdot e^{i( \frac{5\pi}{5} + \frac{\pi}{5})} 
	\]
	
	\hint
	
	Bestem fase
	\[
	= 32 \cdot e^{i \frac{6 \pi}{5}}
	\]
	
	
\end{exercise}

\end{document}
