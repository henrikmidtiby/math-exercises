\documentclass{article}
\usepackage{tekvideoexercises}

\begin{document}
\exercisename{Bestem værdien af bestemte integraler}
\tableofcontents

\begin{exercise}{Bestemte integraler 1-1}

Bestem værdien af det bestemte integrale
\[
I = \int_0^1 x^2 \, dx
\]

$I$ = \answerbox{\frac{1}{3}} 


\hint
Først skal vi finde stamfunktionen til det ubestemte integrale og derefter kan vi indsætte grænserne.

\hint
At integrere er det modsatte af at differentiere.

\hint
Vi skal altså finde en funktion $f(x)$, hvor der gælder at
\[
f'(x) = x^2
\]

\hint
Et polynomie differentieres efter formlen:
\[
\frac{d}{dx} x^n = n \cdot x^{n - 1}
\]

\hint
Løsningen til det bestemte integrale er $f(x) = \frac{1}{3} x^3 + K$.

\hint
Sæt grænserne ind i løsningen til det bestemte integrale
\[
I = \left[ \frac{1}{3} x^3 + K \right]_{x=0}^{x=1}
\]

\hint
Forenkl udtrykket
\[
I = \left( \frac{1}{3} \cdot 1^3 + K \right) - \left( \frac{1}{3} \cdot 0^3 + K \right) = \frac{1}{3}
\]


\end{exercise}

\newpage

\begin{exercise}{Bestemte integraler 1-2}
	
	Bestem værdien af det bestemte integrale
	\[
	I = \int_1^2 x \, dx
	\]
	
	$I$ = \answerbox{\frac{3}{2}} 
	
	
	\hint
	Først skal vi finde stamfunktionen til det ubestemte integrale og derefter kan vi indsætte grænserne.
	
	\hint
	At integrere er det modsatte af at differentiere.
	
	\hint
	Vi skal altså finde en funktion $f(x)$, hvor der gælder at
	\[
	f'(x) = x
	\]
	
	\hint
	Et polynomie differentieres efter formlen:
	\[
	\frac{d}{dx} x^n = n \cdot x^{n - 1}
	\]
	
	\hint
	Løsningen til det bestemte integrale er $f(x) = \frac{1}{2} x^2 + K$.
	
	\hint
	Sæt grænserne ind i løsningen til det bestemte integrale
	\[
	I = \left[ \frac{1}{2} x^2 + K \right]_{x=1}^{x=2}
	\]
	
	\hint
	Forenkl udtrykket
	\[
	I = \left( \frac{1}{2} \cdot 2^2 + K \right) - \left( \frac{1}{2} \cdot 1^2 + K \right) = \frac{4}{2} - \frac{1}{2}=  \frac{3}{2}
	\]

	
\end{exercise}

\newpage

\begin{exercise}{Bestemte integraler 1-3}
	
	Bestem værdien af det bestemte integrale
	\[
	I = \int_0^1 x^4 \, dx
	\]
	
	$I$ = \answerbox{\frac{1}{5}} 
	
	
	\hint
	Først skal vi finde stamfunktionen til det ubestemte integrale og derefter kan vi indsætte grænserne.
	
	\hint
	At integrere er det modsatte af at differentiere.
	
	\hint
	Vi skal altså finde en funktion $f(x)$, hvor der gælder at
	\[
	f'(x) = x^4
	\]
	
	\hint
	Et polynomie differentieres efter formlen:
	\[
	\frac{d}{dx} x^n = n \cdot x^{n - 1}
	\]
	
	\hint
	Løsningen til det bestemte integrale er $f(x) = \frac{1}{5} x^5 + K$.
	
	\hint
	Sæt grænserne ind i løsningen til det bestemte integrale
	\[
	I = \left[ \frac{1}{5} x^5 + K \right]_{x=0}^{x=1}
	\]
	
	\hint
	Forenkl udtrykket
	\[
	I = \left( \frac{1}{5} \cdot 1^5 + K \right) - \left( \frac{1}{5} \cdot 0^5 + K \right) = \frac{1}{5}
	\]
	
	
\end{exercise}

\newpage

\begin{exercise}{Bestemte integraler 1-4}
	
	Bestem værdien af det bestemte integrale
	\[
	I = \int_0^{\pi} \sin(x) \, dx
	\]
	
	$I$ = \answerbox{2} 
	
	
	\hint
	Først skal vi finde stamfunktionen til det ubestemte integrale og derefter kan vi indsætte grænserne.
	
	\hint
	At integrere er det modsatte af at differentiere.
	
	\hint
	Vi skal altså finde en funktion $f(x)$, hvor der gælder at
	\[
	f'(x) = \sin(x)
	\]
	
	\hint
	Ved differentiering af $\cos(x)$ og $\sin(x)$ kan enhedscirklen benyttes. Når der differentieres går man med uret rundt i cirklen - tilsvarende går man mod uret, når der integreres. 
	% Evt. anden forklaring eller angiv blot formlen for differentiering af trigonometriske funktioner?
	\[
	\frac{d}{dx} - \cos(x) = \sin(x)
	\]
	
	\hint
	Løsningen til det bestemte integrale er $f(x) = -\cos(x) + K$.
	
	\hint
	Sæt grænserne ind i løsningen til det bestemte integrale
	\[
	I = \left[- \cos(x) + K \right]_{x=0}^{x=\pi}
	\]
	
	\hint
	Forenkl udtrykket
	\[
	I = \left( - \cos(\pi) + K \right) - \left( - \cos(0) + K \right) = - (-1) + 1 = 2
	\]
	
	
\end{exercise}

\newpage

\begin{exercise}{Bestemte integraler 1-5}
	
		Bestem værdien af det bestemte integrale
	\[
	I = \int_0^{\frac{\pi}{2}} \cos(x) \, dx
	\]
	
	$I$ = \answerbox{1} 
	
	
	\hint
	Først skal vi finde stamfunktionen til det ubestemte integrale og derefter kan vi indsætte grænserne.
	
	\hint
	At integrere er det modsatte af at differentiere.
	
	\hint
	Vi skal altså finde en funktion $f(x)$, hvor der gælder at
	\[
	f'(x) = \cos(x)
	\]
	
	\hint
	Ved differentiering af $\cos(x)$ og $\sin(x)$ kan enhedscirklen benyttes. Når der differentieres går man med uret rundt i cirklen - tilsvarende går man mod uret, når der integreres. 
	% Evt. anden forklaring eller angiv blot formlen for differentiering af trigonometriske funktioner?
	\[
	\frac{d}{dx} \sin(x) = \cos(x)
	\]
	
	\hint
	Løsningen til det bestemte integrale er $f(x) = \sin(x) + K$.
	
	\hint
	Sæt grænserne ind i løsningen til det bestemte integrale
	\[
	I = \left[ \sin(x) + K \right]_{x=0}^{x=\frac{\pi}{2}}
	\]
	
	\hint
	Forenkl udtrykket
	\[
	I = \left( \sin\left(\frac{\pi}{2}\right) + K \right) - \left( \sin(0) + K \right) = 1 - 0 = 1
	\]
	
	
	
\end{exercise}

\newpage

\begin{exercise}{Bestemte integraler 1-6}
	
	Bestem værdien af det bestemte integrale
	\[
	I = \int_0^1 e^x \, dx
	\]
	
	$I$ = \answerbox{e-1} 
	
	
	\hint
	Først skal vi finde stamfunktionen til det ubestemte integrale og derefter kan vi indsætte grænserne.
	
	\hint
	At integrere er det modsatte af at differentiere.
	
	\hint
	Vi skal altså finde en funktion $f(x)$, hvor der gælder at
	\[
	f'(x) = e^x
	\]
	
	\hint
	Den afledte af eksponentiel funktionen $e^x$ er lig eksponentiel funktionen selv. 
	\[
	\frac{d}{dx} e^x= e^x
	\]
	
	\hint
	Løsningen til det bestemte integrale er $f(x) = e^x x^3 + K$.
	
	\hint
	Sæt grænserne ind i løsningen til det bestemte integrale
	\[
	I = \left[ e^x + K \right]_{x=0}^{x=1}
	\]
	
	\hint
	Forenkl udtrykket
	\[
	I = \left( e^1 + K \right) - \left( e^0 + K \right) = e-1
	\]
	
	
\end{exercise}

\newpage

\begin{exercise}{Bestemte integraler 1-7}
	
	Bestem værdien af det bestemte integrale
	\[
	I = \int_0^2 x^2+1 \, dx
	\]
	
	$I$ = \answerbox{\frac{14}{3}} 
	
	
	\hint
	Først skal vi finde stamfunktionen til det ubestemte integrale og derefter kan vi indsætte grænserne.
	
	\hint
	At integrere er det modsatte af at differentiere.
	
	\hint
	
	Ved at benytte sumreglen for integraler kan vi opdele integralet i to.
	\[
	\int x^2+ 1 \, dx = \int x^2 \,dx + \int 1 \, dx
	\]
	\hint
	Vi skal altså først finde en funktion $f(x)$, hvor der gælder at
	\[
	f'(x) = x^2
	\]
	
	\hint
	Et polynomie differentieres efter formlen:
	\[
	\frac{d}{dx} x^n = n \cdot x^{n - 1}
	\]
	
	\hint
	Løsningen til det bestemte integrale er $f(x) = \frac{1}{3} x^3 + K_1$.
		
	\hint
	Vi skal  derefter finde en funktion $f(x)$, hvor der gælder at
	\[
	f'(x) = 1
	\]
	
	\hint
	Et polynomie differentieres efter formlen:
	\[
	\frac{d}{dx} x^n = n \cdot x^{n - 1}
	\]
	
	\hint
	Løsningen til det bestemte integrale er $f(x) = x + K_2$.
	
	
	
	\hint
	Saml de to bestemte integraler og indsæt grænserne i løsningerne
	\[
	I = \left[ \frac{1}{3} x^3 + x + K \right]_{x=0}^{x=2}
	\]
	
	\hint
	Forenkl udtrykket
	\[
	I = \left( \frac{1}{3} \cdot 2^3 + 2+ K \right) - \left( \frac{1}{3} \cdot 0^3 + 0 + K \right) = \frac{8}{3} + 2 = \frac{14}{3}
	\]
	
	
\end{exercise}

\newpage

\begin{exercise}{Bestemte integraler 1-8}
	
	Bestem værdien af det bestemte integrale
	\[
	I = \int_0^2 2x^3 \, dx
	\]
	
	$I$ = \answerbox{8} 
	
	
	\hint
	Først skal vi finde stamfunktionen til det ubestemte integrale og derefter kan vi indsætte grænserne.
	
	\hint
	For at simplificere integralet sættes konstanten 2 uden for integralet
	\[
	= 2 \int x^3 \, dx
	\]
	
	\hint
	At integrere er det modsatte af at differentiere.
	
	\hint
	Vi skal altså finde en funktion $f(x)$, hvor der gælder at
	\[
	f'(x) = x^3
	\]
	
	\hint
	Et polynomie differentieres efter formlen:
	\[
	\frac{d}{dx} x^n = n \cdot x^{n - 1}
	\]
	
	\hint
	Løsningen til det bestemte integrale er $f(x) = \frac{1}{4} x^4 + K$.
	
	\hint
	Sæt grænserne ind i løsningen til det bestemte integrale
	\[
	I = 2 \left[ \frac{1}{4} x^4 + K \right]_{x=0}^{x=2}
	\]
	
	\hint
	Forenkl udtrykket
	\[
	I =2 \left[ \left( \frac{1}{4} \cdot 2^4 + K \right) - \left( \frac{1}{4} \cdot 0^4 + K \right) \right] = 2 \frac{2^4}{4} = \frac{2^4}{2} = 8
	\]
	
	
\end{exercise}

\end{document}
