\documentclass{article}
\usepackage{tekvideoexercises}

\begin{document}
\exercisename{Integration af basis funktioner}
\tableofcontents

\begin{exercise}{Integration af basis funktioner 1-1}

Bestem det værdien af det ubestemte integrale
\[
I = \int x^2 dx
\]
og kald den ukendte konstant for $K$.

$I$ = \answerbox{\frac{1}{3} x^3 + K} 

\hint
At integrere er det modsatte af at differentiere.

\hint
Vi skal altså finde en funktion $f(x)$, hvor der gælder at
\[
f'(x) = x^2
\]

\hint
Et polynomie differentieres efter formlen:
\[
\frac{d}{dx} x^n = n \cdot x^{n - 1}
\]

\hint
Løsningen er $f(x) = \frac{1}{3} x^3 + K$.

\end{exercise}


\begin{exercise}{Integration af basis funktioner 1-2}

Bestem det værdien af det ubestemte integrale
\[
I = \int x dx
\]
og kald den ukendte konstant for $K$.

$I$ = \answerbox{\frac{1}{2} x^2 + K} 

\hint
At integrere er det modsatte af at differentiere.

\hint
Vi skal altså finde en funktion $f(x)$, hvor der gælder at
\[
f'(x) = x
\]

\hint
Et polynomie differentieres efter formlen:
\[
\frac{d}{dx} x^n = n \cdot x^{n - 1}
\]

\hint
Løsningen er $f(x) = \frac{1}{2} x^2 + K$.

\end{exercise}


\begin{exercise}{Integration af basis funktioner 1-3}

Bestem det værdien af det ubestemte integrale
\[
I = \int x^4 dx
\]
og kald den ukendte konstant for $K$.

$I$ = \answerbox{\frac{1}{5} x^5 + K} 

\hint
At integrere er det modsatte af at differentiere.

\hint
Vi skal altså finde en funktion $f(x)$, hvor der gælder at
\[
f'(x) = x^4
\]

\hint
Et polynomie differentieres efter formlen:
\[
\frac{d}{dx} x^n = n \cdot x^{n - 1}
\]

\hint
Løsningen er $f(x) = \frac{1}{5} x^5 + K$.

\end{exercise}


\begin{exercise}{Integration af basis funktioner 1-4}

Bestem det værdien af det ubestemte integrale
\[
I = \int \sin(x) dx
\]
og kald den ukendte konstant for $K$.

$I$ = \answerbox{-\cos(x) + K} 

\hint
At integrere er det modsatte af at differentiere.

\hint
Vi skal altså finde en funktion $f(x)$, hvor der gælder at
\[
f'(x) = \sin(x)
\]

\hint
De afledte af sinus og cosinus er
\[
\frac{d}{dx} \sin(x) = \cos(x)
\qquad \qquad 
\frac{d}{dx} \cos(x) = -\sin(x)
\]


\hint
Løsningen er $f(x) = -\cos(x) + K$.

\end{exercise}


\begin{exercise}{Integration af basis funktioner 1-5}

Bestem det værdien af det ubestemte integrale
\[
I = \int \cos(x) dx
\]
og kald den ukendte konstant for $K$.

$I$ = \answerbox{\sin(x) + K} 

\hint
At integrere er det modsatte af at differentiere.

\hint
Vi skal altså finde en funktion $f(x)$, hvor der gælder at
\[
f'(x) = \cos(x)
\]

\hint
De afledte af sinus og cosinus er
\[
\frac{d}{dx} \sin(x) = \cos(x)
\qquad \qquad 
\frac{d}{dx} \cos(x) = -\sin(x)
\]


\hint
Løsningen er $f(x) = \sin(x) + K$.

\end{exercise}


\begin{exercise}{Integration af basis funktioner 1-6}

Bestem det værdien af det ubestemte integrale
\[
I = \int e^x dx
\]
og kald den ukendte konstant for $K$.

$I$ = \answerbox{e^x + K} 

\hint
At integrere er det modsatte af at differentiere.

\hint
Vi skal altså finde en funktion $f(x)$, hvor der gælder at
\[
f'(x) = e^x
\]

\hint
De afledte af eksponential funktionen er 
\[
\frac{d}{dx} e^x = e^x
\]


\hint
Løsningen er $f(x) = e^x + K$.

\end{exercise}



\begin{exercise}{Integration af basis funktioner 1-7}

Bestem det værdien af det ubestemte integrale
\[
I = \int \frac{1}{x} dx
\]
og kald den ukendte konstant for $K$.

$I$ = \answerbox{\ln(x) + K} 

\hint
At integrere er det modsatte af at differentiere.

\hint
Vi skal altså finde en funktion $f(x)$, hvor der gælder at
\[
f'(x) = \frac{1}{x}
\]

\hint
De afledte af den naturlige logaritme er
\[
\frac{d}{dx} \ln(x) = \frac{1}{x}
\]


\hint
Løsningen er $f(x) = \ln(x) + K$.

\end{exercise}

\end{document}
