\documentclass{article}
\usepackage{tekvideoexercises}

\begin{document}
\exercisename{Bestem determinanter ved hjælp af Gauss elimination}
\tableofcontents
\newpage

\begin{exercise}{Determinanter - 1}

Givet matricen

\[
A = \left[\begin{array}{rr}
1 & 2 \\ 
3 & 5 
\end{array} \right]
\]

Bestem værdien af determinanten vha. Gauss elimination.

$\det(A)$ = \answerbox{-1}

\hint
Benyt Gauss elimination til at omskrive matricen til 
øvre triangulær form.

\hint
Udfør række operationen
$R2 - 3R1 \to R2$.

\hint
Det giver matricen
\[
\left[\begin{array}{rr}
1 & 2 \\ 
0 & -1 
\end{array} \right]
\]

\hint
Nu er matricen på øvre triangulær form.

\hint
Når matricen er på øvre triangulær form, er determinanten 
produktet af diagonal elementerne.

\hint
\[
\det(A) = 1 \cdot (-1) = -1
\]


\end{exercise}


\begin{exercise}{Determinanter - 2}

Givet matricen

\[
A = \left[\begin{array}{rr}
-3 & 4 \\ 
3 & 2 
\end{array} \right]
\]

Bestem værdien af determinanten vha. Gauss elimination.

$\det(A)$ = \answerbox{-18}

\hint
Benyt Gauss elimination til at omskrive matricen til 
øvre triangulær form.

\hint
Udfør række operationen
$R2 + R1 \to R2$.

\hint
Det giver matricen
\[
A = \left[\begin{array}{rr}
-3 & 4 \\ 
0 & 6 
\end{array} \right]
\]

\hint
Nu er matricen på øvre triangulær form.

\hint
Når matricen er på øvre triangulær form, er determinanten 
produktet af diagonal elementerne.

\hint
\[
\det(A) = (-3) \cdot 6 = -18
\]


\end{exercise}


\begin{exercise}{Determinanter - 3}

Givet matricen

\[
A = \left[\begin{array}{rrr}
5 & 2 & 1 \\ 
3 & 3 & 2 \\
1 & 1 & 2
\end{array} \right]
\]

Bestem værdien af determinanten vha. Gauss elimination.

$\det(A)$ = \answerbox{12}

\hint
Benyt Gauss elimination til at omskrive matricen til 
øvre triangulær form.

\hint
Udfør række operationerne
$R1 - 5 R3 \to R1$ og 
$R2 - 3 R3 \to R2$.

\hint
Det giver matricen
\[
\left[\begin{array}{rrr}
0 & -3 & -9 \\ 
0 & 0 & -4 \\
1 & 1 & 2
\end{array} \right]
\]

\hint
Byt række et og række to.

\hint
Det giver matricen
\[
\left[\begin{array}{rrr}
0 & 0 & -4 \\
0 & -3 & -9 \\ 
1 & 1 & 2
\end{array} \right]
\]

\hint 
Byt række et med række tre.

\hint
Det giver matricen
\[
\left[\begin{array}{rrr}
1 & 1 & 2 \\
0 & -3 & -9 \\ 
0 & 0 & -4
\end{array} \right]
\]

\hint
Nu er matricen på øvre triangulær form.

\hint
Når matricen er på øvre triangulær form, er determinanten 
produktet af diagonal elementerne.

\hint
\[
\det(A) = 1 \cdot (-3) \cdot (-4) = 12
\]

\hint
Korriger fortegnet på determinanten ud fra antal række ombytninger.

\hint
Der er byttet rækker to gange, derfor skal fortegnet justeres med
faktoren $(-1)^2 = 1$.
Da faktoren er $1$, vedbliver determinanten at være 12.

\end{exercise}


\begin{exercise}{Determinanter - 4}
	
Givet matricen

\[
A = \left[\begin{array}{rr}
2 & 5 \\ 
1 & 5 
\end{array} \right]
\]
	
Bestem værdien af determinanten vha. Gauss elimination.

$\det(A)$ = \answerbox{5}

\hint
Benyt Gauss elimination til at omskrive matricen til 
øvre triangulær form.

\hint
Udfør række operationen
$R1 - 2 R2 \to R1$.

\hint
Det giver matricen
\[
\left[\begin{array}{rr}
0 & -5 \\ 
1 & 5 
\end{array} \right]
\]
	
\hint
Byt om på række et og række to.

\hint
Det giver matricen
\[
\left[\begin{array}{rr}
1 & 5 \\
0 & -5
\end{array} \right]
\]

\hint
Nu er matricen på øvre triangulær form.

\hint
Når matricen er på øvre triangulær form, er determinanten 
produktet af diagonal elementerne ganget med en korrektionsfaktor.

\hint
Korrektionsfaktoren er bestemt ud fra antal ombytninger af rækker
der er foretaget under gauss eliminationen.
Vi har foretaget en rækkeombytning, faktoren er derfor: $(-1)^1 = -1$.

\hint
\[
\det(A) = 1 \cdot (-5) \cdot (-1) = 5
\]
	
\end{exercise}




\begin{exercise}{Determinanter - 5}
	
Givet matricen
	
\[
A = \left[\begin{array}{rrr}
2 & 3 & 7 \\ 
5 & -2 & 1 \\
3 & 4 & 2
\end{array} \right]
\]
	
Bestem værdien af determinanten vha. Gauss elimination.

$\det(A)$ = \answerbox{145}

\hint
Benyt Gauss elimination til at omskrive matricen til 
øvre triangulær form.

\hint
Udfør række operationen
$R1 - R3 \to R1$.

\hint
Det giver matricen
\[
\left[\begin{array}{rrr}
-1 & -1 & 5 \\ 
5 & -2 & 1 \\
3 & 4 & 2
\end{array} \right]
\]
	
\hint
Udfør række operationerne
$R2 + 5R1 \to R2$ og $R3 + 3R1 \to R3$.

\hint
Det giver matricen
\[
\left[\begin{array}{rrr}
-1 & -1 & 5 \\ 
0 & -7 & 26 \\
0 & 1 & 17
\end{array} \right]
\]
	
\hint
Udfør række operationerne
$R2 + 7R3 \to R2$.

\hint
Det giver matricen
\[
\left[\begin{array}{rrr}
-1 & -1 & 5 \\ 
0 & 0 & 145 \\
0 & 1 & 17
\end{array} \right]
\]
	
\hint
Byt om på række to og række tre.

\hint
Det giver matricen
\[
\left[\begin{array}{rrr}
-1 & -1 & 5 \\ 
0 & 1 & 17 \\
0 & 0 & 145
\end{array} \right]
\]

\hint
Nu er matricen på øvre triangulær form.

\hint
Når matricen er på øvre triangulær form, er determinanten 
produktet af diagonal elementerne ganget med en korrektionsfaktor.

\hint
Korrektionsfaktoren er bestemt ud fra antal ombytninger af rækker
der er foretaget under gauss eliminationen.
Vi har foretaget en rækkeombytning, faktoren er derfor: $(-1)^1 = -1$.

\hint
\[
\det(A) = (-1) \cdot 1 \cdot 145 \cdot (-1) = 145
\]
	
\end{exercise}





\begin{exercise}{Determinanter og Gauss elimination - 1}
	
Givet matricen

\[
A = \left[\begin{array}{rrr}
-2 & 1 & 4 \\ 
7 & 8 & -1 \\
5 & -4 & 2
\end{array} \right]
\]

Bestem værdien af determinanten vha. Gauss elimination.

$\det(A)$ = \answerbox{-315}


\hint
\[
\det(A) = \left|\begin{array}{rrr}
-2 & 1 & 4 \\ 
7 & 8 & -1 \\
5 & -4 & 2
\end{array} \right|
\]


\hint
Udfør række operationen
$R3 + 2R1 \to R3$.

\hint
Det giver 
\[
\left|\begin{array}{rrr}
-2 & 1 & 4 \\ 
7 & 8 & -1 \\
5 & -4 & 2
\end{array} \right|
=
\left|\begin{array}{rrr}
-2 & 1 & 4 \\ 
7 & 8 & -1 \\
1 & -2 & 10
\end{array} \right|
\]
	
\hint
Udfør række operationerne
$R1 + 2R3 \to R1$ og $R2 - 7R3 \to R2$.

\hint
Det giver matricen
\[
\left|\begin{array}{rrr}
-2 & 1 & 4 \\ 
7 & 8 & -1 \\
1 & -2 & 10
\end{array} \right| = 
\left|\begin{array}{rrr}
0 & -3 & 24 \\ 
0 & 22 & -71 \\
1 & -2 & 10
\end{array} \right|
\]
	
\hint
Udfør række operationerne
$\frac{-1}{3} R1 \to R1$.
Det ændrer determinanten med en faktor $\frac{-1}{3}$, 
dette korrigeres der for ved at gange med $-3$.

\hint
Det giver 
\[
\left|\begin{array}{rrr}
0 & -3 & 24 \\ 
0 & 22 & -71 \\
1 & -2 & 10
\end{array} \right| = 
\left|\begin{array}{rrr}
0 & 1 & -8 \\ 
0 & 22 & -71 \\
1 & -2 & 10
\end{array} \right| \cdot (-3)
\]
	
\hint
Udfør række operationerne
$R2 - 22R1 \to R2$.

\hint
Det giver matricen
\[
\left|\begin{array}{rrr}
0 & 1 & -8 \\ 
0 & 22 & -71 \\
1 & -2 & 10
\end{array} \right| \cdot (-3) = 
\left|\begin{array}{rrr}
0 & 1 & -8 \\ 
0 & 0 & 105 \\
1 & -2 & 10
\end{array} \right| \cdot (-3)
\]

\hint
Byt om på række to og række tre, det skifter fortegnet på determinanten.
Korriger derfor for dette ved at gange med -1.

\hint
Det giver 
\[
\left|\begin{array}{rrr}
0 & 1 & -8 \\ 
0 & 0 & 105 \\
1 & -2 & 10
\end{array} \right| \cdot (-3) = 
\left|\begin{array}{rrr}
0 & 1 & -8 \\ 
1 & -2 & 10 \\
0 & 0 & 105
\end{array} \right| \cdot 3
\]

\hint
Byt om på række et og række to, det skifter fortegnet på determinanten.
Korriger derfor for dette ved at gange med -1.

\hint
Det giver matricen
\[
\left|\begin{array}{rrr}
0 & 1 & -8 \\ 
1 & -2 & 10 \\
0 & 0 & 105
\end{array} \right| \cdot 3 = 
\left|\begin{array}{rrr}
1 & -2 & 10 \\
0 & 1 & -8 \\ 
0 & 0 & 105
\end{array} \right| \cdot (-3)
\]

\hint
Nu er matricen på øvre triangulær form.

\hint
Når matricen er på øvre triangulær form, er determinanten 
produktet af diagonal elementerne.

\hint
\[
\left|\begin{array}{rrr}
1 & -2 & 10 \\
0 & 1 & -8 \\ 
0 & 0 & 105
\end{array} \right| \cdot (-3)
= 1 \cdot 1 \cdot 105 \cdot (-3)
= -315
\]
	
\end{exercise}


\begin{exercise}{Determinanter og Gauss elimination - 2}
	
Givet matricen
	
\[
A = \left[\begin{array}{rrr}
4 & 5 & 9 \\ 
3 & 4 & 8 \\
-2 & 1 & 0
\end{array} \right]
\]

Bestem værdien af determinanten vha. Gauss elimination.

$\det(A)$ = \answerbox{-13}


\hint
\[
\det(A) = \left|\begin{array}{rrr}
4 & 5 & 9 \\ 
3 & 4 & 8 \\
-2 & 1 & 0
\end{array} \right|
\]


\hint
Udfør række operationen
$R1 - R2 \to R1$.

\hint
Det giver 
\[
\left|\begin{array}{rrr}
4 & 5 & 9 \\ 
3 & 4 & 8 \\
-2 & 1 & 0
\end{array} \right|
=
\left|\begin{array}{rrr}
1 & 1 & 1 \\ 
3 & 4 & 8 \\
-2 & 1 & 0
\end{array} \right|
\]
	
\hint
Udfør række operationerne
$R2 - 3R1 \to R2$ og $R3 + 2R1 \to R3$.

\hint
Det giver
\[
\left|\begin{array}{rrr}
1 & 1 & 1 \\ 
3 & 4 & 8 \\
-2 & 1 & 0
\end{array} \right|
=
\left|\begin{array}{rrr}
1 & 1 & 1 \\ 
0 & 1 & 5 \\
0 & 3 & 2
\end{array} \right|
\]

\hint
Udfør rækkeoperationen
$R3 - 3R2 \to R3$.

\hint
Det giver
\[
\left|\begin{array}{rrr}
1 & 1 & 1 \\ 
0 & 1 & 5 \\
0 & 3 & 2
\end{array} \right|
=
\left|\begin{array}{rrr}
1 & 1 & 1 \\ 
0 & 1 & 5 \\
0 & 0 & -13
\end{array} \right|
\]

\hint
Matricen er nu på øvre triangulær form.

\hint
\[
\left|\begin{array}{rrr}
1 & 1 & 1 \\ 
0 & 1 & 5 \\
0 & 0 & -13
\end{array} \right|
= 1 \cdot 1 \cdot (-13) = -13
\]
	
\end{exercise}


\begin{exercise}{Determinanter og Gauss elimination - 3}
	
Givet matricen

\[
A = \left[\begin{array}{rrr}
-2 & 5 & 8 \\ 
9 & -3 & 2 \\
1 & 6 & 7
\end{array} \right]
\]

Bestem værdien af determinanten vha. Gauss elimination.

$\det(A)$ = \answerbox{217}


\hint
\[
\det(A) = \left|\begin{array}{rrr}
-2 & 5 & 8 \\ 
9 & -3 & 2 \\
1 & 6 & 7
\end{array} \right|
\]


\hint
Udfør række operationerne
$R1 + 2R3 \to R1$ og
$R2 - 9R3 \to R2$.

\hint
Det giver 
\[
\left|\begin{array}{rrr}
-2 & 5 & 8 \\ 
9 & -3 & 2 \\
1 & 6 & 7
\end{array} \right|
=
\left|\begin{array}{rrr}
0 & 17 & 22 \\ 
0 & -57 & -61 \\
1 & 6 & 7
\end{array} \right|
\]

\hint
Udfør række operationen
$R2 + 3R1 \to R2$.

\hint
Det giver 
\[
\left|\begin{array}{rrr}
0 & 17 & 22 \\ 
0 & -57 & -61 \\
1 & 6 & 7
\end{array} \right|
=
\left|\begin{array}{rrr}
0 & 17 & 22 \\ 
0 & -6 & 5 \\
1 & 6 & 7
\end{array} \right|
\]

\hint
Udfør række operationen
$R1 + 3R2 \to R1$.

\hint
Det giver 
\[
\left|\begin{array}{rrr}
0 & 17 & 22 \\ 
0 & -6 & 5 \\
1 & 6 & 7
\end{array} \right|
=
\left|\begin{array}{rrr}
0 & -1 & 37 \\ 
0 & -6 & 5 \\
1 & 6 & 7
\end{array} \right|
\]

\hint
Udfør række operationen
$R2 - 6R1 \to R2$.

\hint
Det giver 
\[
\left|\begin{array}{rrr}
0 & -1 & 37 \\ 
0 & -6 & 5 \\
1 & 6 & 7
\end{array} \right|
=
\left|\begin{array}{rrr}
0 & -1 & 37 \\ 
0 & 0 & -217 \\
1 & 6 & 7
\end{array} \right|
\]

\hint
Byt række et og række to.
Korriger derefter skiftet i fortegn på determinanten ved at
gange med (-1).

\hint
Det giver 
\[
\left|\begin{array}{rrr}
0 & -1 & 37 \\ 
0 & 0 & -217 \\
1 & 6 & 7
\end{array} \right|
=
\left|\begin{array}{rrr}
0 & 0 & -217 \\
0 & -1 & 37 \\ 
1 & 6 & 7
\end{array} \right| \cdot (-1)
\]

\hint
Byt række et og række tre.
Korriger derefter skiftet i fortegn på determinanten ved at
gange med (-1).

\hint
Det giver 
\[
\left|\begin{array}{rrr}
0 & 0 & -217 \\
0 & -1 & 37 \\ 
1 & 6 & 7
\end{array} \right| \cdot (-1)
=
\left|\begin{array}{rrr}
1 & 6 & 7 \\
0 & -1 & 37 \\ 
0 & 0 & -217
\end{array} \right|
\]

\hint
Matricen er nu på øvre triangulær form.

\hint
\[
\left|\begin{array}{rrr}
1 & 6 & 7 \\
0 & -1 & 37 \\ 
0 & 0 & -217
\end{array} \right|
= 1 \cdot (-1) \cdot (-217) = -217
\]


\end{exercise}




\end{document}
