\documentclass{article}
\usepackage{tekvideoexercises}

\begin{document}
\exercisename{Grænseværdier basis viden}
\tableofcontents
\newpage

\begin{exercise}{Grænseværdier basis viden 1}

Bestem \[\lim_{x \to 2} = 3x^4\]


\answerbox{48}

\hint
Indsæt $x=2$ i udtrykket og se om det giver mening.

\hint
\[
\lim_{x \to 2} = 3x^4 = 3 \cdot 2^4
\]

\hint 
Det er der ikke problemer ved.

\hint
\[
3 \cdot 2^4 = 48
\]

\end{exercise}

\newpage
\begin{exercise}{Grænseværdier basis viden 2}
	
	Bestem \[\lim_{x \to -2} = \frac{x-2}{x+5}\]
	
	
	\answerbox{\frac{-4}{3}}
	
	\hint
	Indsæt $x=-2$ i udtrykket og se om det giver mening.
	
	\hint
	\[
	\lim_{x \to -2} = \frac{-2-2}{-2+5}
	\]
	
	\hint 
	Det er der ikke problemer ved.
	
	\hint
	\[
	\frac{-2-2}{-2+5} = \frac{-4}{3}
	\]
	
\end{exercise}

\newpage
\begin{exercise}{Grænseværdier basis viden 3}
	
	Bestem \[\lim_{x \to 3} = \frac{x+4}{x-4}\]
	
	
	\answerbox{-7}
	
	\hint
	Indsæt $x=3$ i udtrykket og se om det giver mening.
	
	\hint
	\[
	\lim_{x \to 3} = \frac{x+4}{x-4} = \frac{3+4}{3-4}
	\]
	
	\hint 
	Det er der ikke problemer ved.
	
	\hint
	\[
	\frac{3+4}{3-4} = \frac{7}{-1} = -7
	\]
	
\end{exercise}

\newpage
\begin{exercise}{Grænseværdier basis viden 4}
	
	Bestem \[\lim_{x \to 2} = \frac{x^2+1}{x+3}\]
	
	
	\answerbox{1}
	
	\hint
	Indsæt $x=2$ i udtrykket og se om det giver mening.
	
	\hint
	\[
	\lim_{x \to 2} = \frac{x^2+1}{x+3} = \frac{2^2+1}{2+3}
	\]
	
	\hint 
	Det er der ikke problemer ved.
	
	\hint
	\[
	\frac{2^2+1}{2+3} = \frac{5}{5} = 1
	\]
	
\end{exercise}

\newpage
\begin{exercise}{Grænseværdier basis viden 5}
	
	Bestem \[\lim_{x \to -1} = \frac{x^2}{1+x^2}\]
	
	
	\answerbox{\frac{1}{2}}
	
	\hint
	Indsæt $x=-1$ i udtrykket og se om det giver mening.
	
	\hint
	\[
	\lim_{x \to -1} = \frac{x^2}{1+x^2} = \frac{(-1)^2}{1 + (-1)^2}
	\]
	
	\hint 
	Det er der ikke problemer ved.
	
	\hint
	\[
	\frac{(-1)^2}{1 + (-1)^2} = \frac{1}{1+1} = \frac{1}{2}
	\]
	
\end{exercise}

\newpage
\begin{exercise}{Grænseværdier basis viden 6}
	
	Bestem \[\lim_{x \to -1} = \frac{x+2}{x^3-1}\]
	
	
	\answerbox{\frac{-1}{2}}
	
	\hint
	Indsæt $x=-1$ i udtrykket og se om det giver mening.
	
	\hint
	\[
	\lim_{x \to -1} = \frac{x+2}{x^3-1} = \frac{-1+2}{(-1)^3 -1}
	\]
	
	\hint 
	Det er der ikke problemer ved.
	
	\hint
	\[
	 \frac{-1+2}{(-1)^3 -1} = \frac{1}{-1-1} = \frac{-1}{2}
	\]
	
\end{exercise}

\newpage
\begin{exercise}{Grænseværdier basis viden 7}
	
	Bestem \[\lim_{x \to 5} = \frac{x^2-20}{2x+3}\]
	
	
	\answerbox{\frac{5}{13}}
	
	\hint
	Indsæt $x=5$ i udtrykket og se om det giver mening.
	
	\hint
	\[
	\lim_{x \to 5} = \frac{x^2-20}{2x+3} = \frac{5^2-20}{2 \cdot 5 +3}
	\]
	
	\hint 
	Det er der ikke problemer ved.
	
	\hint
	\[
	\frac{5^2-20}{2 \cdot 5 +3} = \frac{25-20}{10+3} = \frac{5}{13}	
	\]
	
\end{exercise}

\newpage
\begin{exercise}{Grænseværdier basis viden 8}
	
	Bestem \[\lim_{x \to 0} = \frac{x^2+1}{x+2}\]
	
	
	\answerbox{\frac{1}{2}}
	
	\hint
	Indsæt $x=0$ i udtrykket og se om det giver mening.
	
	\hint
	\[
	\lim_{x \to 0} = \frac{x^2+1}{x+2} = \frac{0^2+1}{0+2}
	\]
	
	\hint 
	Det er der ikke problemer ved.
	
	\hint
	\[
	 \frac{0^2+1}{0+2} = \frac{1}{2}
	\]
	
\end{exercise}

\newpage


\end{document}
