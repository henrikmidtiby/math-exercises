\documentclass{article}
\usepackage{tekvideoexercises}

\begin{document}
\exercisename{Logaritme regneregler}
\tableofcontents
\newpage

\begin{exercise}{Logaritme regneregler 1}

Bestem $\log_{27}(3)$.

\answerbox{1/3}

\hint

Anvend at logaritmer er det modsatte af eksponenter
\[
\log_a(b) = c \qquad \Leftrightarrow \qquad a^c = b
\]

\hint

Indsæt $a=27$, $b=3$ og $c=x$.

\hint
\[
27^x = 3
\]

\hint
Tag den $x$'te rod

\hint
\[
27 = 3^{1/x}
\]

\hint
Hvor mange gange skal 3 ganges med sig selv for at få 27?

\hint
Tre gange
\[
3 \cdot 3 \cdot 3 = 27
\]

\hint
Derfor må $1/x = 3$.

\hint
Svaret er $\log_{27}(3) = 1/3$.

\end{exercise}

\newpage

\begin{exercise}{Logaritme regneregler 2}
	
	Bestem $\log_{32}(2)$.
	
	\answerbox{1/5}
	
	\hint
	
	Anvend at logaritmer er det modsatte af eksponenter
	\[
	\log_a(b) = c \qquad \Leftrightarrow \qquad a^c = b
	\]
	
	\hint
	
	Indsæt $a=32$, $b=2$ og $c=x$.
	
	\hint
	\[
	32^x = 2
	\]
	
	\hint
	Tag den $x$'te rod
	
	\hint
	\[
	32 = 2^{1/x}
	\]
	
	\hint
	Hvor mange gange skal 2 ganges med sig selv for at få 32?
	
	\hint
	Fem gange
	\[
	2 \cdot 2 \cdot 2 \cdot 2 \cdot 2 = 32
	\]
	
	\hint
	Derfor må $1/x = 5$.
	
	\hint
	Svaret er $\log_{32}(2) = 1/5$.
	
\end{exercise}

\newpage
\begin{exercise}{Logaritme regneregler 3}
	
	Bestem $\log_{81}(3)$.
	
	\answerbox{1/4}
	
	\hint
	
	Anvend at logaritmer er det modsatte af eksponenter
	\[
	\log_a(b) = c \qquad \Leftrightarrow \qquad a^c = b
	\]
	
	\hint
	
	Indsæt $a=81$, $b=3$ og $c=x$.
	
	\hint
	\[
	81^x = 3
	\]
	
	\hint
	Tag den $x$'te rod
	
	\hint
	\[
	81 = 3^{1/x}
	\]
	
	\hint
	Hvor mange gange skal 3 ganges med sig selv for at få 81?
	
	\hint
	Fire gange
	\[
	3 \cdot 3 \cdot 3 \cdot 3 = 81
	\]
	
	\hint
	Derfor må $1/x = 4$.
	
	\hint
	Svaret er $\log_{81}(3) = 1/4$.
	
\end{exercise}

\newpage
\begin{exercise}{Logaritme regneregler 4}
	
	Bestem $\log_{5}(1/5)$.
	
	\answerbox{-1}
	
	\hint
	
	Anvend at logaritmer er det modsatte af eksponenter
	\[
	\log_a(b) = c \qquad \Leftrightarrow \qquad a^c = b
	\]
	
	\hint
	
	Indsæt $a=5$, $b=1/5$ og $c=x$.
	
	\hint
	\[
	5^x = 1/5
	\]
	
	\hint 
	
	Benyt at $1/x = x^{-1}$

	\hint
	\[
	5^x = 5^{-1}
	\]

	\hint
	
	Sammenlign eksponenter
	
	\hint 
	\[
	x = -1
	\]
	
	\hint
	Svaret er $\log_{5}(1/5) = -1$
	
	
\end{exercise}

\newpage
\begin{exercise}{Logaritme regneregler 5}
	
	Bestem $\log_{10}(2) + \log_{10}(5)$.
	
	\answerbox{1}
	
	\hint
	
	Benyt logaritmeregnereglen $\log(a \cdot b) =  \log(a) + \log(b)$

	\hint
	\[
	\log_{10}(2) + \log_{10}(5) = \log_{10}(2 \cdot 5) = \log_{10}(10)
	\]
	
	\hint 
	Benyt at $\log_{x}(x) = 1$
	
	\hint
	
	Svaret er $\log_{10}(2) + \log_{10}(5) = 1$
\end{exercise}

\newpage
\begin{exercise}{Logaritme regneregler 6}
	
	Bestem $\log_{2}(2/16) $.
	
	\answerbox{-3}
	
	\hint
	
	Benyt logaritmeregnereglen $\log(a / b) =  \log(a) - \log(b)$
	
	\hint
	\[
	\log_{2}(16/2) = \log_{2}(2) -  \log_{2}(16)
	\]
	
	\hint 
	Benyt at $\log_{x}(x) = 1$
	
	\hint
	\[
	1 - \log_{2}(16)
	\]
	
	\hint
	Benyt at $2 \cdot 2 \cdot 2 \cdot 2 = 2^4 = 16$
	
	\hint
	\[
	1 - \log_{2}(16) = 1 - \log_{2}(2^4)
	\]
	
	\hint
	Benyt logaritmeregnereglen $\log(a^x) = x \cdot \log(a)$
		
	\hint
	\[	 
	= 1- 4 \cdot \log_{2}(2)
	\]
	
	\hint 
	Benyt at $\log_{x}(x) = 1$
	
	\hint
	
	Svaret er $\log_{2}(2/16) = -3 $
\end{exercise}

\newpage
\begin{exercise}{Logaritme regneregler 7}
	
	Bestem $\log_{y}(2y) $.
	
	\answerbox{\log_{y}(2)  +1}
	
	\hint
	
	Benyt logaritmeregnereglen $\log(a \cdot  b) =  \log(a) + \log(b)$
	
	\hint
	\[
	\log_{y}(2y) = \log_{y}(2) + \log_{y}(y)
	\]
	
	\hint 
	Benyt at $\log_{x}(x) = 1$
	
	\hint
	
	Svaret er $\log_{y}(2y) = \log_{y}(2) + 1 $
\end{exercise}

\newpage
\begin{exercise}{Logaritme regneregler 8}
	
	Bestem $\log_{6}(12) + \log_{6}(3)$.
	
	\answerbox{2}
	
	\hint
	
	Benyt logaritmeregnereglen $\log(a \cdot b) =  \log(a) + \log(b)$
	
	\hint
	\[
	\log_{6}(12) + \log_{6}(3) = \log_{6}(12 \cdot 3) = \log_{6}(36)
	\]
	
	\hint
	Benyt at $6 \cdot 6  = 36$
		
	\hint 
	\[
	\log_{6}(36) = \log_{6}(6^2)
	\]
	
	\hint
	Benyt logaritmeregnereglen $\log(a^x) = x \cdot \log(a)$

	\hint
	\[
	 \log_{6}(6^2) = 2 \cdot \log_{6}(6)
	\]
	
	\hint
	Benyt $\log_{x}(x) = 1$
	
	\hint
	
	Svaret er $\log_{6}(12) + \log_{6}(3) = 2$
\end{exercise}

\end{document}
