\documentclass{article}
\usepackage[utf8]{inputenc}
\usepackage{todonotes}
\usepackage{graphicx}
\usepackage{amsmath}
\usepackage[colorlinks, linkcolor=blue, citecolor=blue, urlcolor=blue]{hyperref}


\newenvironment{exercise}[1]{\newpage\section{#1}}{}

\newcommand{\answerbox}[1]{\fbox{$#1$}}

\newcommand{\hint}{\subsection*{Hint}}

\begin{document}
Anvend startbetingelser
\tableofcontents

\newpage

\begin{exercise}{Separabel differentialligning 2-1}

	
	Vi har løst en differentialligning og fundet den generelle løsning
	\[
	y(x) = \frac{1}{2}x^2 + A
	\]
	Bestem konstanten A givet startbetingelsen $y(2)=7$.
	
	$A$ = \answerbox{5}
	
	\hint
	
	Indsæt startbetingelse i den generelle løsning. 

	
	\hint
	
	Vi indsætter først $x=2$.
	
	\hint
	
	Det giver
	\[
	y(2)= \frac{1}{2}(2)^2+A = 2 + A
	\]
	
	\hint
	
	Derefter indsættes $y(2)=7$.

	\hint 
	
	Det giver 
		\[
	7 = 2 + A \Rightarrow A = 5
	\]
	
\end{exercise}

\newpage

\begin{exercise}{Separabel differentialligning 2-2}

	
	Vi har løst en differentialligning og fundet den generelle løsning
	\[
	y(x) = \sqrt{x^2 + A}
	\]
	Bestem konstanten A givet startbetingelsen $y(2)=5$.
	
	$A$ = \answerbox{21}
	
	\hint
	
	Indsæt startbetingelse i den generelle løsning. 
	
	
	\hint
	
	Vi indsætter først $x=2$.
	
	\hint
	
	Det giver
	\[
	y(2)= \sqrt{2^2 + A} = \sqrt{4 + A}
	\]
	
	\hint
	
	Derefter indsættes $y(2)=5$.
	
	\hint 
	
	Det giver 
	\[
	5 = \sqrt{4 + A} \Rightarrow 25 = 4 + A \Leftrightarrow A = 21	
	\]
	
\end{exercise}

\newpage

\begin{exercise}{Separabel differentialligning 2-3}

	Vi har løst en differentialligning og fundet den generelle løsning
	\[
	y(x) = - \cos(x) + A
	\]
	Bestem konstanten A givet startbetingelsen $y(\pi)=5$.
	
	$A$ = \answerbox{4}
	
	\hint
	
	Indsæt startbetingelse i den generelle løsning. 
	
	
	\hint
	
	Vi indsætter først $x=\pi$.
	
	\hint
	
	Det giver
	\[
	y(\pi)= -\cos (\pi) + A = 1 + A
	\]
	
	\hint
	
	Derefter indsættes $y(\pi)=5$.
	
	\hint 
	
	Det giver 
	\[
	5 = 1 + A  \Rightarrow A = 4
	\]
	
\end{exercise}

\newpage


\begin{exercise}{Separabel differentialligning 2-4}

	
	Vi har løst en differentialligning og fundet den generelle løsning
	\[
	y(x) =  A \cdot e^{5x}
	\]
	Bestem konstanten A givet startbetingelsen $y(0)=7$.
	
	$A$ = \answerbox{7}
	
	\hint
	
	Indsæt startbetingelse i den generelle løsning. 
	
	
	\hint
	
	Vi indsætter først $x=0$.
	
	\hint
	
	Det giver
	\[
	y(0)=  A \cdot e^{5 \cdot 0} = A \cdot e^0  = A
	\]
	
	\hint
	
	Derefter indsættes $y(0)=7$.
	
	\hint 
	
	Det giver 
	\[
	7  = A  
	\]
	
\end{exercise}

\newpage


\begin{exercise}{Separabel differentialligning 1-5}
	
	
	Vi har løst en differentialligning og fundet den generelle løsning
	\[
	y(x) =  A \cdot \sqrt{x}
	\]
	Bestem konstanten A givet startbetingelsen $y(9)=12$, hvor $A>0$.
	
	$A$ = \answerbox{4}
	
	\hint
	
	Indsæt startbetingelse i den generelle løsning. 
	
	
	\hint
	
	Vi indsætter først $x=9$.
	
	\hint
	
	Det giver
	\[
	y(9)=  A \cdot \sqrt{9} = A	\cdot 3
	\]
	
	\hint
	
	Derefter indsættes $y(9)=12$.
	
	\hint 
	
	Det giver 
	\[
	12  = 3A  \Rightarrow A = 4
	\]
	
\end{exercise}

\end{document}

