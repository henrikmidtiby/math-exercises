\documentclass{article}
\usepackage[utf8]{inputenc}
\usepackage{todonotes}
\usepackage{graphicx}
\usepackage{amsmath}
\usepackage[colorlinks, linkcolor=blue, citecolor=blue, urlcolor=blue]{hyperref}


\newenvironment{exercise}[1]{\newpage\section{#1}}{}

\newcommand{\answerbox}[1]{\fbox{$#1$}}

\newcommand{\hint}{\subsection*{Hint}}

\begin{document}
\tableofcontents
\newpage

\begin{exercise}{Separabel differentialligning 1-1}


Find den generelle løsning til differentialligningen 
\[
\frac{dy}{dx} = x 
\]
Kald den ukendte konstant for A.

$\textrm{Generel løsning } y(x)$ = \answerbox{\frac{1}{2}x^2 + A}

\hint

Først konstateres det at ligningen er separabel, idet den er på formen
\[
\frac{dy}{dx} = f(x) \cdot g(y)
\]


\hint

Integrer på begge sider af lighedstegnet mht. x
\[
\int\frac{dy}{dx}\, dx= \int x\,dx
\]

\hint

$dx$'erne går ud med hinanden på venstresiden.
\[
\int\ dy\,= \int x\,dx
\]

\hint

Integralerne løses
\[
y + K_1 = \frac{1}{2} x^2 + K_2
\]


\hint

Konstanter samles i $A$
\[
y = \frac{1}{2}x^2 + A
\]
hvor $A=K_2-K_1$

\end{exercise}


\newpage

\begin{exercise}{Separabel differentialligning 1-2}

	
	Find den generelle løsning til differentialligningen 
	\[
	\frac{dy}{dx} = x  \cdot \frac{1}{y}
	\]
	Se på tilfældet hvor $y$ er positivt og kald den ukendte konstant for A.
	
	$\textrm{Generel løsning } y(x)$ = \answerbox{\sqrt{x^2 + A}}
	
	\hint
	
	Først konstateres det at ligningen er separabel, idet den er på formen
	\[
	\frac{dy}{dx} = f(x) \cdot g(y)
	\]
	
	\hint 
	Der ganges igennem med $y$ på begge sider af lighedstegnet
	\[
	y \frac{dy}{dx} = x
	\]
	
	\hint
	
	Integrer på begge sider af lighedstegnet mht. x
	\[
	\int\ y \frac{dy}{dx}\, dx= \int x\,dx
	\]
	
	\hint
	
	$dx$'erne går ud med hinanden på venstresiden.
	\[
	\int\ y \, dy= \int x\,dx
	\]
	
	\hint
	
	Integralerne løses
	\[
	\frac{1}{2} y^2 + K_1 = \frac{1}{2} x^2 + K_2
	\]
	
	
	\hint
	
	Der ganges igennem med 2, og konstanter samles i $A$
	\[
	y^2 = x^2 + A
	\]
	hvor $A=2(K_2-K_1)$
	
	\hint
	
	$y$ isoleres ved at tage kvadratroden af begge sider
	\[
	y = \pm \sqrt{x^2 + A}
	\]

	\hint
	
	Da vi kun er interesserede i den positive løsning, har vi resultatet:
	\[
	y = \sqrt{x^2 + A}
	\]
	
\end{exercise}

\newpage

\begin{exercise}{Separabel differentialligning 1-3}


	Find den generelle løsning til differentialligningen 
	\[
	\frac{dy}{dx} = \sin(x)
	\]
	Kald den ukendte konstant for A.
	
	$\textrm{Generel løsning } y(x)$ = \answerbox{- \cos(x) + A}
	
	\hint
	
	Først konstateres det at ligningen er separabel, idet den er på formen
	\[
	\frac{dy}{dx} = f(x) \cdot g(y)
	\]
	
	\hint
	
	Integrer på begge sider af lighedstegnet mht. x
	\[
	\int\ \frac{dy}{dx}\, dx= \int \sin (x)\,dx
	\]
	
	\hint
	
	$dx$'erne går ud med hinanden på venstresiden.
	\[
	\int\ \, dy= \int \sin(x)\,dx
	\]
	
	\hint
	
	Integralerne løses
	\[
	y + K_1 =- \cos (x) + K_2
	\]
	
	
	\hint
	
	Konstanter samles i A
	\[
	y = - \cos(x) + A
	\]
	hvor $A=K_2-K_1$
	
\end{exercise}


\newpage

\begin{exercise}{Separabel differentialligning 1-4}

	
	Find den generelle løsning til differentialligningen 
	\[
	\frac{dy}{dx} = 5y
	\]
	Kald den ukendte konstant for A.
	
	$\textrm{Generel løsning } y(x)$ = \answerbox{A \cdot e^{5x}}
	
	\hint
	
	Først konstateres det at ligningen er separabel, idet den er på formen
	\[
	\frac{dy}{dx} = f(x) \cdot g(y)
	\]
	
	\hint 
	Der divideres med y på begge sider af lighedstegnet
	\[
	\frac{1}{y} \frac{dy}{dx} = 5
	\]
	
	\hint
	
	Integrer på begge sider af lighedstegnet mht. x
	\[
	\int\ \frac{1}{y} \frac{dy}{dx} \, dx= \int 5 \,dx
	\]
	
	\hint
	
	$dx$'erne går ud med hinanden på venstresiden og konstanten 5 trækkes uden for integrationstegnet
	\[
	\int\ \frac{1}{y}\, dy= 5 \cdot \int \,dx
	\]
	
	\hint
	
	Integralerne løses
	\[
	\ln(y) + K_1 =5x + K_2
	\]
	
	
	\hint
	
	Konstanter samles i $K_3$
	\[
	\ln(y) =5x + K_3
	\]
	hvor $K_3=K_2-K_1$
	
	\hint 
	
	Vi opløfter begge sider i $e$ for at slippe af med $\ln(y)$
	\[
	y = e^{5x+K_3}
	\]
	
	\hint
	
	Vha.  af potensregneregler omskrives udtrykket
	\[
	y = e^{5x} \cdot e^{K_3}
	\]
	
	\hint
	
	Konstanter samles i $A$
	\[
	y = A \cdot e^{5x}
	\]
	hvor $ A= e^{K_3}$
	
	\end{exercise}

\newpage

\begin{exercise}{Separabel differentialligning 1-5}
	
	
	Find den generelle løsning til differentialligningen 
	\[
	\frac{dy}{dx} = \frac{y}{2x}
	\]
	Kald den ukendte konstant for A.
	
	$\textrm{Generel løsning } y(x)$ = \answerbox{A \cdot \sqrt{x}}
	
	\hint
	
	Først konstateres det at ligningen er separabel, idet den er på formen
	\[
	\frac{dy}{dx} = f(x) \cdot g(y)
	\]
	
	\hint 
	Der divideres med y på begge sider af lighedstegnet
	\[
	\frac{1}{y} \frac{dy}{dx} = \frac{1}{2x}
	\]
	
	\hint
	
	Integrer på begge sider af lighedstegnet mht. x
	\[
	\int\ \frac{1}{y} \frac{dy}{dx} \, dx= \int \frac{1}{2x} \,dx
	\]
	
	\hint
	
	$dx$'erne går ud med hinanden på venstresiden og konstanten $\frac{1}{2}$ trækkes uden for integrationstegnet
	\[
	\int\ \frac{1}{y}\, dy= 5 \cdot \int \frac{1}{x}\,dx
	\]
	
	\hint
	
	Integralerne løses
	\[
	\ln(y) + K_1 =\frac{1}{2} \ln(x) + K_2
	\]
	
	
	\hint
	
	Konstanter samles i $K_3$
	\[
	\ln(y) =\frac{1}{2} \ln(x) + K_3
	\]
	hvor $K_3=K_2-K_1$
	
	\hint 
	
	Vi opløfter begge sider i $e$ for at slippe af med $\ln(y)$
	\[
	y = e^{\frac{1}{2} \ln(x) + K_3}
	\]
	
	\hint
	
	Vha.  af potensregneregler omskrives udtrykket
	\[
	y = e^{\frac{1}{2} \ln(x) + K_3} = e^{\frac{1}{2} \ln(x)}  \cdot e^{K_3}
	\]
	
	\hint
	
	Konstanter samles i $A$
	\[
	y = A \cdot e^{\frac{1}{2} \ln(x)}
	\]
	hvor $ A= e^{K_3}$
	
	\hint 
	
	Udtrykket omskrives vha. logaritmeregler
	\[
	y = A \cdot e^{\frac{1}{2} \ln(x)} = A \cdot e^{\ln(x^{\frac{1}{2})}} = A \cdot x^{\frac{1}{2}} = A \cdot \sqrt{x}
	\]
	
\end{exercise}

\end{document}

