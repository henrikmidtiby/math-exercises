\documentclass{article}
\usepackage[utf8]{inputenc}
\usepackage{todonotes}
\usepackage{graphicx}
\usepackage{amsmath}
\usepackage[colorlinks, linkcolor=blue, citecolor=blue, urlcolor=blue]{hyperref}

\newenvironment{exercise}[1]{\newpage\section{#1}}{}
\newcommand{\answerbox}[1]{\fbox{$#1$}}
\newcommand{\hint}{\subsection*{Hint}}

\begin{document}
\tableofcontents
\newpage

\begin{exercise}{Potensregneregler 2-1}

Bestem $2^{-6} \cdot 4^4$.

\answerbox{4}

\hint 
Omskriv udtrykket
\[
2^{-6} \cdot 4^4 = 2^{-6} \cdot (2 \cdot 2)^4
\]

\hint

Benyt potensregnereglen $(x \cdot y)^n = x^n \cdot y^n$
\[
2^{-6} \cdot (2 \cdot 2)^4 = 2^{-6} \cdot 2^4 \cdot 2^4
\]

\hint
Benyt potensregnereglen $ x^n \cdot x^m  = x^{n+m}$
\[
2^{-6} \cdot 2^4 \cdot 2^4 = 2^{-6+4+4} = 2^2  = 4
\]

\end{exercise}

\newpage

\begin{exercise}{Potensregneregler 2-2}
	
	Bestem $a^{-5} \cdot a^{-3}$.
	
	\answerbox{a^{-8}}
	
	\hint
	
	Benyt potensregnereglen 
	\[
	x^n \cdot x^m  = x^{n+m}
	\] 
	\hint 
	
	Sæt $x=a$, $n=-5$ og $m=-3$.
	\[
	a^{-5} \cdot a^{-3} = a^{-5-3} = a^{-8}
	\]
	
\end{exercise}

\newpage

\begin{exercise}{Potensregneregler 2-3}
	
	Bestem $a^{3} \cdot a^{9}$.
	
	\answerbox{a^{-6}}
	
	\hint
	Benyt potensregnereglen 
	\[
	x^n \cdot x^m  = x^{n+m}
	\] 
	
	\hint 
	
	Sæt $x=a$, $n=3$ og $m=-9$.
	\[
	a^{3} \cdot a^{-9} = a^{3-9} = a^{-6}
	\]
	
\end{exercise}

\newpage

\begin{exercise}{Potensregneregler 2-4}
	
	Bestem $\frac{1}{a^{-2}} \cdot a^{2}$.
	
	\answerbox{a^{4}}
	
	\hint
	Benyt potensregnereglen 
	\[
	 x^{-n} = \frac{1}{x^{n}} 
	\]
	
	\hint 
	
	Omskriv udtrykket
	\[
	\frac{1}{a^{-2}} \cdot a^{2} = a^{-(-2)} \cdot a^2 	= a^2 \cdot a^2
	\]
	
	\hint
	Benyt potensregnereglen 
	\[
	x^n \cdot x^m  = x^{n+m}
	\]
	
	\hint 
	
	Sæt $x=a$, $n=2$ og $m=2$.
	\[
	a^{2} \cdot a^{2} = a^{2+2} = a^{4}
	\]
	
\end{exercise}

\newpage

\begin{exercise}{Potensregneregler 2-5}
	
	Bestem $-2^{-4}$.
	
	\answerbox{\frac{1}{16}}
	
	\hint
	Benyt potensregnereglen 
	\[
	x^{-n} = \frac{1}{x^{n}} 
	\]
	
	\hint 
	
	Sæt $x=-2$ og $n=4$ 
	\[
	-2^{-4} = \frac{1}{-2^4} = \frac{1}{16}
	\]
	
\end{exercise}

\newpage

\begin{exercise}{Potensregneregler 2-6}
	
	Bestem $3^{-2}$.
	
	\answerbox{\frac{1}{9}}
	
	\hint
	Benyt potensregnereglen 
	\[
	x^{-n} = \frac{1}{x^{n}} 
	\]
	
	\hint 
	
	Sæt $x=3$ og $n=2$ 
	\[
	3^{-2} = \frac{1}{3^2} = \frac{1}{9}
	\]
	
\end{exercise}

\newpage

\begin{exercise}{Potensregneregler 2-7}
	
	Bestem $4 \cdot (a \cdot b)^{-2} \cdot a^2$
		
	\answerbox{4 \cdot b^{-2}}
	
	\hint
	
	Benyt potensregnereglen $(x \cdot y)^n = x^n \cdot y^n$
	\[
	4 \cdot (a \cdot b)^{-2} \cdot a^2 = 4 \cdot a^{-2} \cdot b^{-2} \cdot a^{2}
	\]
	
	\hint
	Benyt potensregnereglen 
	\[
	x^n \cdot x^m  = x^{n+m} 
	\]
	
	\hint 
	
	Reducer udtryk
	\[
	4 \cdot a^{-2} \cdot b^{-2} \cdot a^{2} = 4 \cdot a^{-2+2} \cdot b^{-2} = 4 \cdot b^{-2}
	\]
	
\end{exercise}

\newpage

\begin{exercise}{Potensregneregler 2-8}
	
	Bestem $(4 \cdot x \cdot y^2)^2 \cdot (x \cdot y)^{-2}$
	
	\answerbox{16 \cdot y^2}
	
	\hint
	
	Benyt potensregnereglen $(x \cdot y)^n = x^n \cdot y^n$
	\[
	(4 \cdot x \cdot y^2)^2 \cdot (x \cdot y)^{-2} = 4^2 \cdot x^2 \cdot {y^2}^2 \cdot x^{-2} \cdot y^{-2}
	\]
		
	\hint
	Benyt potensregnereglen 
	\[
	(x^n)^m = x^{n \cdot m}
	\]
	
	\hint 
	Reducer udtryk vha. potensregnereglen $x^n \cdot x^m  = x^{n+m}$
	\[
	16 \cdot x^2 \cdot y^{2\cdot 2} \cdot x^{-2} \cdot y^{-2} = 16 \cdot x^{2-2} \cdot y^{4-2}
	\]
	
	\hint 
	
	Reducer udtryk
	\[
	16 \cdot x^{2-2} \cdot y^{4-2} = 16 \cdot y^2
	\]
	
\end{exercise}


\end{document}
