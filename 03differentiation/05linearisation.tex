\documentclass{article}
\usepackage{tekvideoexercises}

\begin{document}
\tableofcontents
\newpage


\begin{exercise}{Linearisering 1}
% Adams 4.9.7
Lineariser nedenstående funktion omkring punktet $x = \pi$.
\[
\sin(x)
\]

\answerbox{\pi - x}

\hint
Lineariseringen af en funktion $f(x)$
omkring punktet $x_0$ er givet ved
\[
f(x_0) + f'(x_0) \cdot (x - x_0)
\]

\hint
Vi anvender $f(x) = \sin(x)$ og $x_0 = \pi$.

\hint
$f'(x)$ bestemmes
\[
f'(x) = \cos(x)
\]

\hint
Værdierne fra opgaven indsættes i udtrykket for lineariseringen.
\[
\sin(\pi) + \cos(\pi) \cdot (x - \pi)
\]

\hint
Der forsimples
\[
\sin(\pi) = 0 \qquad 
\cos(\pi) = -1
\]
og indsættes i udtrykket
\[
(0) + (-1) \cdot (x - \pi) = 
\pi - x
\]

\end{exercise}

\begin{exercise}{Linearisering 2}
% Adams 4.9.19
Benyt en passende linearisering til at approksimere
\[
\cos(49^\circ)
\]
angiv resultatet med fire decimalaer i boksen herunder.
\answerbox{0.6577}

\hint
Først skal vi vælge den funktion der skal approksimeres
og det punkt som der skal bruges som udgangspunkt for 
approksimationen.
For at vi kan bestemme approksimationen skal vi 
kende funktionsværdien og dens afledte i punktet.

\hint
Vi vælger funktionen $f(x) = \cos(x)$ og at approksimere den 
omkring $x_0 = 45^\circ = \pi / 4$.
Så kan vi nemlig godt finde værdierne af $f(\pi / 4)$ og $f'(\pi / 4)$.
\[
f(\pi / 4) = \cos(\pi / 4) = \frac{1}{\sqrt{2}}
\]
\[
f'(\pi / 4) = -\sin(\pi / 4) = \frac{-1}{\sqrt{2}}
\]

\hint
Lineariseringen opskrives
\[
L(x) = \frac{1}{\sqrt{2}} - \frac{1}{\sqrt{2}} \cdot (x - \pi / 4)
\]

\hint
Værdien i punktet bestemmes
\[
L(49 \cdot \pi / 180) = \frac{1}{\sqrt{2}} - \frac{1}{\sqrt{2}} \cdot (49 \cdot \pi / 180 - \pi / 4) 
\]
det forsimples til 
\[
\frac{1}{\sqrt{2}} - \frac{1}{\sqrt{2}} \cdot \pi / 45
\]
som vha. lommeregner bestemmes til 
\[
0.657741 \simeq 0.6577
\]

\hint
For at vurdere fejlen på approksimationen, ser på anden 
ordens taylor udviklingen (en orden højere end den 
anvendte linearisering).
\[
f(x_0) 
	+ \frac{f'(x_0)}{1!} \cdot (x - x_0) 
	+ \frac{f''(x_0)}{2!} \cdot (x - x_0)^2
\]
Fejlen er relateret til det sidste led i det ovenstående udtryk.
Hvor $f''(x_0)$ udskiftes med $f''(c)$ hvor $c \in [x_0; x]$.
\[
\textrm{fejl} = \frac{f''(c)}{2!} \cdot (x - x_0)^2
\]

\hint 
Antager man at $c = \pi/4$, fåes følgende fejlvurdering
\[
\textrm{fejl} = \frac{f''(\pi/4)}{2!} \cdot (49 \cdot \pi / 180 - \pi / 4)^2 = 
-0.00172318\]

\hint
Sammenligning mellem reel værdi, approksimeret værdi og estimeret maksimal fejl.
\[
\cos(49 \cdot \pi/180) = 0.656059
\]
\[
L(49 \cdot \pi/180) = 0.657741
\]
\[
\cos(49 \cdot \pi/180) - L(49 \cdot \pi/180) = -0.001682
\]

\end{exercise}

\begin{exercise}{Linearisering 3}

	Benyt en passende linearisering til at approksimere
	\[
	\sqrt{50}
	\]
	angiv resultatet med fire decimalaer i boksen herunder.
	\answerbox{7.0714}
	
	\hint
	Først skal vi vælge den funktion der skal approksimeres
	og det punkt som der skal bruges som udgangspunkt for 
	approksimationen.
	For at vi kan bestemme approksimationen skal vi 
	kende funktionsværdien og dens afledte i punktet.
	
	\hint
	Vi vælger funktionen $f(x) = \sqrt{x}$ og at approksimere den 
	omkring $x_0 = 49$.
	Så kan vi nemlig godt finde værdierne af $f(49)$ og $f'(49)$.
	\[
	f(49) = \sqrt{49} = 7
	\]
	\[
	f'(49) = \frac{1}{2 \sqrt{49}} = \frac{1}{2 \cdot 7 } = \frac{1}{14}
	\]
	
	\hint
	Lineariseringen opskrives
	\[
	L(x) = 7 + \frac{1}{14} \cdot (x - 49) = \frac{x + 49}{14}
	\]
	
	\hint
	Værdien i punktet bestemmes
	\[
	L(50) = \frac{50 + 49}{14} = \frac{99}{14}
	\]
	som vha. lommeregner bestemmes til 
	\[
	7.071428 \simeq 7.0714
	\]
	
	\hint
	For at vurdere fejlen på approksimationen, ser vi på anden 
	ordens taylor udviklingen (en orden højere end den 
	anvendte linearisering).
	\[
	f(x_0) 
	+ \frac{f'(x_0)}{1!} \cdot (x - x_0) 
	+ \frac{f''(x_0)}{2!} \cdot (x - x_0)^2
	\]
	Fejlen er relateret til det sidste led i det ovenstående udtryk.
	Hvor $f''(x_0)$ udskiftes med $f''(c)$ hvor $c \in [x_0; x]$.
	\[
	\textrm{fejl} = \frac{f''(c)}{2!} \cdot (x - x_0)^2
	\]
	
	\hint 
	Antager man at $c = 49$, fåes følgende fejlvurdering
	\[
	\textrm{fejl} = \frac{f''(49)}{2!} \cdot (50 - 49)^2 = 
	-0.0003644\]
	
	\hint
	Sammenligning mellem reel værdi, approksimeret værdi og estimeret maksimal fejl.
	\[
	\sqrt{50}= 7.071067
	\]
	\[
	L(50) = 7.071428
	\]
	\[
	\sqrt{50} - L(50)  = -0.000361
	\]
	
\end{exercise}

\begin{exercise}{Linearisering 4}
	
	Benyt en passende linearisering til at approksimere
	\[
	\sqrt{17}
	\]
	angiv resultatet med fire decimalaer i boksen herunder.
	\answerbox{4.1250}
	
	\hint
	Først skal vi vælge den funktion der skal approksimeres
	og det punkt som der skal bruges som udgangspunkt for 
	approksimationen.
	For at vi kan bestemme approksimationen skal vi 
	kende funktionsværdien og dens afledte i punktet.
	
	\hint
	Vi vælger funktionen $f(x) = \sqrt{x}$ og at approksimere den 
	omkring $x_0 = 16$.
	Så kan vi nemlig godt finde værdierne af $f(16)$ og $f'(16)$.
	\[
	f(16) = \sqrt{16} = 4
	\]
	\[
	f'(16) = \frac{1}{2 \sqrt{16}} = \frac{1}{2 \cdot 4 } = \frac{1}{8}
	\]
	
	\hint
	Lineariseringen opskrives
	\[
	L(x) = 4 + \frac{1}{8} \cdot (x - 16) = \frac{x + 16}{8}
	\]
	
	\hint
	Værdien i punktet bestemmes
	\[
	L(17) = \frac{17 + 16}{8} = \frac{33}{8}
	\]
	som vha. lommeregner bestemmes til 
	\[
	4.125000 \simeq 4.1250
	\]
	
	\hint
	For at vurdere fejlen på approksimationen, ser vi på anden 
	ordens taylor udviklingen (en orden højere end den 
	anvendte linearisering).
	\[
	f(x_0) 
	+ \frac{f'(x_0)}{1!} \cdot (x - x_0) 
	+ \frac{f''(x_0)}{2!} \cdot (x - x_0)^2
	\]
	Fejlen er relateret til det sidste led i det ovenstående udtryk.
	Hvor $f''(x_0)$ udskiftes med $f''(c)$ hvor $c \in [x_0; x]$.
	\[
	\textrm{fejl} = \frac{f''(c)}{2!} \cdot (x - x_0)^2
	\]
	
	\hint 
	Antager man at $c = 16$, fåes følgende fejlvurdering
	\[
	\textrm{fejl} = \frac{f''(16)}{2!} \cdot (17 - 16)^2 = 
	-0.0019531\]
	
	\hint
	Sammenligning mellem reel værdi, approksimeret værdi og estimeret maksimal fejl.
	\[
	\sqrt{17}= 4.123106
	\]
	\[
	L(17) = 4.125000
	\]
	\[
	\sqrt{17} - L(17)  = -0.001894
	\]
	
\end{exercise}

\begin{exercise}{Linearisering 5}
	
	Benyt en passende linearisering til at approksimere
	\[
	\sqrt{37}
	\]
	angiv resultatet med fire decimalaer i boksen herunder.
	\answerbox{6.0833}
	
	\hint
	Først skal vi vælge den funktion der skal approksimeres
	og det punkt som der skal bruges som udgangspunkt for 
	approksimationen.
	For at vi kan bestemme approksimationen skal vi 
	kende funktionsværdien og dens afledte i punktet.
	
	\hint
	Vi vælger funktionen $f(x) = \sqrt{x}$ og at approksimere den 
	omkring $x_0 = 36$.
	Så kan vi nemlig godt finde værdierne af $f(36)$ og $f'(36)$.
	\[
	f(36) = \sqrt{36} = 6
	\]
	\[
	f'(36) = \frac{1}{2 \sqrt{36}} = \frac{1}{2 \cdot 6 } = \frac{1}{12}
	\]
	
	\hint
	Lineariseringen opskrives
	\[
	L(x) = 6 + \frac{1}{12} \cdot (x - 36) = \frac{x + 36}{12}
	\]
	
	\hint
	Værdien i punktet bestemmes
	\[
	L(37) = \frac{37 + 36}{12} = \frac{73}{12}
	\]
	som vha. lommeregner bestemmes til 
	\[
	6.083333 \simeq 6.0833
	\]
	
	\hint

	For at vurdere fejlen på approksimationen, ser vi på anden 
	ordens taylor udviklingen (en orden højere end den 
	anvendte linearisering).
	\[
	f(x_0) 
	+ \frac{f'(x_0)}{1!} \cdot (x - x_0) 
	+ \frac{f''(x_0)}{2!} \cdot (x - x_0)^2
	\]
	Fejlen er relateret til det sidste led i det ovenstående udtryk.
	Hvor $f''(x_0)$ udskiftes med $f''(c)$ hvor $c \in [x_0; x]$.
	\[
	\textrm{fejl} = \frac{f''(c)}{2!} \cdot (x - x_0)^2
	\]
	
	\hint 
	Antager man at $c = 36$, fåes følgende fejlvurdering
	\[
	\textrm{fejl} = \frac{f''(36)}{2!} \cdot (37 - 36)^2 = 
	-0.0005780
	\]
	
	\hint
	Sammenligning mellem reel værdi, approksimeret værdi og estimeret maksimal fejl.
	\[
	\sqrt{37}= 6.0827625
	\]
	\[
	L(37) = 6.083333
	\]
	\[
	\sqrt{37} - L(37)  = -0.0005708
	\]
	
\end{exercise}

\begin{exercise}{Linearisering 6}

	Lineariser nedenstående funktion omkring punktet $x = 5$.
	\[
	\sqrt{9-x}
	\]
	
	\answerbox{\frac{13 - x}{4}}
	
	\hint
	Lineariseringen af en funktion $f(x)$
	omkring punktet $x_0$ er givet ved
	\[
	f(x_0) + f'(x_0) \cdot (x - x_0)
	\]
	
	\hint
	Vi anvender $f(x) = \sqrt{9-x}$ og $x_0 = 5$.
	
	\hint
	$f'(x)$ bestemmes
	\[
	f'(x) = \frac{-1}{2 \sqrt{9 - x}}
	\]
	
	\hint
	Værdierne fra opgaven indsættes i udtrykket for lineariseringen.
	\[
	\sqrt{9 - 5} - \frac{1}{2 \sqrt{9 - 5}} \cdot (x - 5) = \sqrt{4} - \frac{(x-5)}{2 \sqrt{4}} 
	\]
	
	\hint
	Der reduceres
	\[
	= 2 + \frac{-x + 5}{4} = \frac{13 - x}{4}
	\]
	
\end{exercise}

\begin{exercise}{Linearisering 7}
	
	Lineariser nedenstående funktion omkring punktet $x = 3$.
	\[
	\frac{1}{(1+x)^2}
	\]
	
	\answerbox{\frac{5 - x}{32}}
	
	\hint
	Lineariseringen af en funktion $f(x)$
	omkring punktet $x_0$ er givet ved
	\[
	f(x_0) + f'(x_0) \cdot (x - x_0)
	\]
	
	\hint
	Vi anvender $f(x) = \frac{1}{(1+x)^2}$ og $x_0 = 3$.
	
	\hint
	$f'(x)$ bestemmes
	\[
	f'(x) = \frac{-2}{(1+x)^3}
	\]
	
	\hint
	Værdierne fra opgaven indsættes i udtrykket for lineariseringen.
	\[
	\frac{1}{(1+3)^2} - \frac{2}{(1+3)^3} \cdot (x-3)^2 = \frac{1}{16} - \frac{2 \cdot (x - 3)}{64}
	\]
	
	\hint
	Der reduceres
	\[
	= \frac{1}{16} + \frac{6 - 2x}{64} = \frac{10 - 2x}{64} = \frac{5-x}{32}
	\]
	
\end{exercise}

\begin{exercise}{Linearisering 8}
	
	Lineariser nedenstående funktion omkring punktet $x = \frac{1}{2}$.
	\[
	e^{2x}
	\]
	
	\answerbox{2x \cdot e}
	
	\hint
	Lineariseringen af en funktion $f(x)$
	omkring punktet $x_0$ er givet ved
	\[
	f(x_0) + f'(x_0) \cdot (x - x_0)
	\]
	
	\hint
	Vi anvender $f(x) = e^{2x}$ og $x_0 = \frac{1}{2}$.
	
	\hint
	$f'(x)$ bestemmes
	\[
	f'(x) = 2 e^{2x}
	\]
	
	\hint
	Værdierne fra opgaven indsættes i udtrykket for lineariseringen.
	\[
	e^{2 \cdot \frac{1}{2}} + 2 e^{2 \cdot \frac{1}{2}} \cdot (x - \frac{1}{2}) = e + 2 e \cdot (x - \frac{1}{2})
	\]
	
	\hint
	Der reduceres
	\[
	= e + 2x \cdot e - e = 2x \cdot e
	\]
	
\end{exercise}


\end{document}
